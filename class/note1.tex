\chapter{Función Gamma (Factorial)}
\section{Definición}
\dfn{Definición}{
	La función gamma de Euler se define para \( \operatorname{Re}(z) > 0 \) por la integral impropia:

	\[
		\Gamma(z) := \int_0^\infty t^{z-1} e^{-t} \, \dd{t}
	\]

	Es una extensión del factorial, pues para \( n \in \mathbb{N} \):

	\[
		\Gamma(n) = (n-1)!
	\]

}

También puede ser equivalente a:
$$
	\Gamma(z) := x^z \int_0^\infty t^{z-1} e^{-kt} \, \dd{t}, \quad (\Re z >0, \Re k >0)
$$


\mprop{}{	Valor en 1: \( \Gamma(1) = 1 \).}

\begin{proof}
	Usamos la definición de la función gamma de Euler, sustituyendo $z = 1$, se tiene:
	\[
		\Gamma(1) = \int_0^{\infty} t^{1-1} e^{-t} \, \dd{t} = \int_0^{\infty} e^{-t} \, \dd{t} = \left[ -e^{-t} \right]_0^{\infty} = 0 - (-1) = 1.
	\]
	Por lo tanto,
	\[
		\Gamma(1) = 1. \qedhere
	\]
\end{proof}

\mprop{Relación funcional de la función Gamma}{
	$$\Gamma(z+1) = z \, \Gamma(z) $$
}


\begin{proof}
	Usamos la definición:
	\[
		\Gamma(z+1) = \int_0^\infty t^z e^{-t} \, \dd{t}.
	\]
	Aplicamos integración por partes con:
	\[
		u = t^z \Rightarrow du = z t^{z-1} \dd{t}, \quad dv = e^{-t} \dd{t} \Rightarrow v = -e^{-t}.
	\]
	Entonces:
	\[
		\Gamma(z+1) = \left[ -t^z e^{-t} \right]_0^\infty + \int_0^\infty z t^{z-1} e^{-t} \, \dd{t}.
	\]
	El término evaluado en los extremos es cero, ya que:
	\[
		\lim_{t \to \infty} t^z e^{-t} = 0 \quad \text{y} \quad \lim_{t \to 0^+} t^z e^{-t} = 0 \quad \text{para } \operatorname{Re}(z) > 0.
	\]
	Por tanto,
	\[
		\Gamma(z+1) = z \int_0^\infty t^{z-1} e^{-t} \, \dd{t} = z \, \Gamma(z). \qedhere
	\]
\end{proof}

\nt{De la recurrencia,
	\begin{align*}
		\Gamma(n+z) & = (n-1+z)(n-2+z) \ldots (1+z) \Gamma(1+z) \\
		            & =(n-1+z)(n-2+z) \ldots (1+z) z!           \\
		            & = (n-1+z)!
	\end{align*}}
\ex{Encuentre el valor de $\Gamma(\frac{5}{2})$ si $\Gamma(\frac{1}{2}) = \sqrt{\pi}$}{

	\begin{align*}
		\Gamma\qty(\frac{5}{2}) & = \Gamma\qty(1+\frac{3}{2})  =  \frac{3}{2} \cdot \Gamma\qty(\frac{3}{2}) =  \frac{3}{2} \cdot\pqty{ \Gamma\qty(1+\frac{1}{2}) } = \\
		                        & =  \frac{3}{2} \cdot\pqty{\frac{1}{2}\cdot \Gamma\qty(\frac{1}{2}) } = \frac{3}{4} \cdot \Gamma\qty(\frac{1}{2})                   \\
		                        & = \frac{3}{4} \sqrt{\pi}
	\end{align*}
}


\mprop{Producto de Weierstrass}{
	La función gamma admite la representación infinita:

	\[
		\frac{1}{\Gamma(z)} = z e^{\gamma z} \prod_{n=1}^\infty \left(1 + \frac{z}{n} \right) e^{-z/n}, \quad z \in \mathbb{C} \setminus \{0, -1, -2, \dots\}
	\]

	donde la constante \( \gamma \), llamada \textbf{constante de Euler–Mascheroni}, se define como:

	\[
		\gamma := \lim_{m \to \infty} \left( \sum_{k=1}^m \frac{1}{k} - \ln m \right) \approx 0.5772\ 1566\ 4900\ldots
	\]
}

\mprop{Fórmula de reflexión de Euler}{
	Para todo \( z \notin \mathbb{Z} \), se cumple:
	\[
		\Gamma(z) \Gamma(1 - z) = \frac{\pi}{\sin(\pi z)}.
	\]}

\begin{proof}
	Utilizamos la fórmula del producto de Weierstrass para la función Gamma:
	\[
		\frac{1}{\Gamma(z)} = z e^{\gamma z} \prod_{n=1}^\infty \left(1 + \frac{z}{n} \right) e^{-z/n},
	\]

	Entonces:
	\[
		\Gamma(z) = \frac{1}{z} e^{-\gamma z} \prod_{n=1}^\infty \left(1 + \frac{z}{n} \right)^{-1} e^{z/n}.
	\]

	Análogamente:
	\[
		\Gamma(1 - z) = \frac{1}{1 - z} e^{-\gamma(1 - z)} \prod_{n=1}^\infty \left(1 + \frac{1 - z}{n} \right)^{-1} e^{(1 - z)/n}.
	\]

	Además \( \pi / \sin(\pi z) \), es igual a:
	\[
		\frac{\pi}{\sin(\pi z)} = \frac{1}{z} \prod_{n=1}^\infty \left(1 - \frac{z^2}{n^2} \right)^{-1}.
	\]

	\[
		\Rightarrow \Gamma(z)\Gamma(1 - z) = \frac{\pi}{\sin(\pi z)}.
		\qedhere
	\]
\end{proof}

\ex{$\Gamma(\frac{1}{2}) = \sqrt{\pi}$}{
	Haciendo $z = \frac{1}{2}$ en la reflexión de Euler
	\begin{align*}
		\Gamma\qty(\frac{1}{2}) \Gamma\qty(1-\frac{1}{2}) & = \frac{\pi}{\sin(\pi \cdot \frac{1}{2})} \\
		\pqty{\Gamma\qty(\frac{1}{2})}^2                  & = \pi                                     \\
		\Gamma\qty(\frac{1}{2})                           & = \sqrt{\pi}
	\end{align*}
}

\section{ Otras Propiedades Fundamentales}

\begin{itemize}


	\item \textbf{Fórmula de Stirling (asimptótica):}
	      \[
		      \Gamma(z) \sim \sqrt{2\pi} z^{z - 1/2} e^{-z} \left( 1 + \frac{1}{12z} + \frac{1}{288z^2} - \cdots \right)
	      \]
	\item \textbf{Fórmula de duplicación de Legendre:}
	      \[
		      \Gamma(2z) = \frac{\Gamma(z)\Gamma\left(z + \tfrac{1}{2} \right)}{2^{1 - 2z} \sqrt{\pi}}
	      \]

	\item \textbf{Fórmula de triplicación:}

	      \[
		      \Gamma(3z) = \frac{3^{3z - \frac{1}{2}}}{2\pi} \, \Gamma(z) \, \Gamma\left(z + \frac{1}{3} \right) \, \Gamma\left(z + \frac{2}{3} \right)
	      \]

	\item \textbf{Fórmula de multiplicación de Gauss:}

	      Para \( n \in \mathbb{N} \), se tiene:

	      \[
		      \Gamma(nz) = (2\pi)^{\frac{1-n}{2}} \, n^{nz - \frac{1}{2}} \prod_{k=0}^{n-1} \Gamma\left(z + \frac{k}{n} \right)
	      \]

	\item \textbf{Euler's Formula:}
	      \[
		      \Gamma(z) = \lim_{n \to \infty} \frac{n! \, n^z}{z (z+1) (z+2) \cdots (z+n)} = \lim_{n \to \infty} \frac{n! \, n^z}{\prod_{k=0}^{n} (z + k)}
		      \quad \text{para } z \notin \{0, -1, -2, \dots\}
	      \]


	\item \textbf{Coeficiente binomial generalizado:}

	      \[
		      \binom{z}{w} = \frac{\Gamma(z+1)}{\Gamma(w+1)\Gamma(z-w+1)}
		      \quad \text{para } z, w \in \mathbb{C} \setminus \mathbb{Z}_{< -1}
	      \]

	\item \textbf{Propiedades en el plano complejo:}

	      \begin{align*}
		      \Gamma(\overline{z})     & = \overline{\Gamma(z)}                                        \\
		      \ln \Gamma(\overline{z}) & = \overline{\ln \Gamma(z)}                                    \\
		      |\Gamma(x + i y)|        & \leq |\Gamma(x)| \quad \text{para } x > 0,\, y \in \mathbb{R} \\
		      \arg \Gamma(z+1)         & = \arg \Gamma(z) + \arg z
	      \end{align*}

	\item \textbf{Serie de potencias de \( \ln \Gamma(1+z) \):}

	      Para \( |z| < 1 \):

	      \[
		      \ln \Gamma(1 + z) = -\gamma z + \sum_{n=2}^\infty \frac{\zeta(n)}{n} (-z)^n
	      \]

	      donde \( \zeta(n) \) es la función zeta de Riemann.

	\item \textbf{Expansión en serie de \( \dfrac{1}{\Gamma(z)} \):}

	      Hay una expansión entera (ver proposición de Weierstrass),  también existe una expansión de tipo:

	      \[
		      \frac{1}{\Gamma(z)} = \sum_{n=1}^\infty c_n z^n, \quad \text{con } c_n \in \mathbb{R}
		      \quad \text{(convergente en } \mathbb{C})
	      \]

	\item \textbf{Fórmula de Stirling:}

	      Para \( |z| \to \infty \), \( |\arg z| < \pi \):

	      \[
		      \Gamma(z) \sim \sqrt{2\pi} \, z^{z - \frac{1}{2}} e^{-z} \left( 1 + \frac{1}{12z} + \frac{1}{288z^2} - \frac{139}{51840z^3} + \cdots \right)
	      \]
	\item \textbf{Desarrollo asintótico de \( \ln \Gamma(z) \):}

	      \[
		      \ln \Gamma(z) \sim z \ln z - z + \frac{1}{2} \ln(2\pi) + \sum_{n=1}^\infty \frac{B_{2n}}{2n(2n-1) z^{2n-1}}
		      \quad \text{para } |z| \to \infty,\, |\arg z| < \pi
	      \]

	      donde \( B_{2n} \) son los números de Bernoulli.



\end{itemize}

\section{Ejercicios }

\subsection{Ejercicios Resueltos}
\begin{enumerate}
	\item Demostrar:
	      \[
		      \Gamma\left(\frac{1+i}{2}\right) \Gamma\left(\frac{1-i}{2}\right)= \sqrt{\pi} \sech(\frac{\pi}{2})
	      \]
\end{enumerate}


\subsection{Ejercicios Propuestos}

\begin{enumerate}

	\item Evaluar la siguiente integral,
	      \[
		      \int_0^\infty x^n e^{-\lambda x} \, dx
		      \quad \textbf{Respuesta: } \frac{\Gamma(n+1)}{\lambda^{n+1}}
	      \]

	\item Evaluar:
	      \[
		      \int_0^1 (-\ln t)^s \, \dd{t}
		      \quad \textbf{Respuesta: } \Gamma(s+1)
	      \]

	\item Demostrar que:
	      \[
		      \int_0^\infty e^{-x^2} \, dx = \frac{\sqrt{\pi}}{2}
		      \quad \text{usando } \Gamma\left(\frac{1}{2}\right)
	      \]


	\item Mostrar que:
	      \[
		      \int_0^\infty \frac{t^{z-1}}{1 + t} \, \dd{t} = \pi / \sin(\pi z)
		      \quad \text{para } 0 < \operatorname{Re}(z) < 1
	      \]

\end{enumerate}

\section{Función Beta}
