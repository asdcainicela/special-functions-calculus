\chapter{Función Gamma (Factorial)}
\section{Definición}
\dfn{Definición}{
	La función gamma de Euler se define   por la integral impropia. 	Si $\Re(z)>-1$ entonces, la Función Gamma converge absolutamente:

	\[
		\Gamma(z) := \int_0^\infty t^{z-1} e^{-t} \, \dd{t}
	\]

	$$
		\Gamma(z+1)=\int_0^{\infty} t^{z } e^{-t}  \dd{t} , \quad z \neq -1,-2,-3,\ldots
	$$

	Es una extensión del factorial, pues para \( n \in \mathbb{N} \):

	\[
		\Gamma(n) = (n-1)!
	\]

}

También puede ser equivalente a:
$$
	\Gamma(z) := x^{-z} \int_0^\infty t^{z-1} e^{-kt} \, \dd{t}, \quad (\Re z >0, \Re k >0)
$$
$$
\Gamma(n) = \int_0^1 \pqty{-\ln(t)}^{z-1} \dd{t}
$$

\ex{}{
	$$
		\int_{0}^{\infty} t^{i+1}  e^{-t}dt=\int_{0}^{\infty} t^{(i+2)-1} e^{-t}dt=\Gamma(i+2)
	$$
	$$
		\int_{0}^{\infty} t^{2} e^{-t}dt=\int_{0}^{\infty} t^{3-1} e^{-t}dt=\Gamma(3)=2!=2
	$$
	$$
		\int_{0}^{\infty} t^{5/9} e^{-t}dt=\int_{0}^{\infty} t^{14/9-1} e^{-t}dt=\Gamma\left(\frac{14}{9} \right)
	$$
}
\mprop{}{	Valor en 1: \( \Gamma(1) = 1 \).}

\begin{proof}
	Usamos la definición de la función gamma de Euler, sustituyendo $z = 1$, se tiene:
	\[
		\Gamma(1) = \int_0^{\infty} t^{1-1} e^{-t} \, \dd{t} = \int_0^{\infty} e^{-t} \, \dd{t} = \left[ -e^{-t} \right]_0^{\infty} = 0 - (-1) = 1.
	\]
	Por lo tanto,
	\[
		\Gamma(1) = 1. \qedhere
	\]
\end{proof}

\mprop{Relación funcional de la función Gamma}{
	$$\Gamma(z+1) = z \, \Gamma(z) $$
}

\begin{proof}
	Usamos la definición:
	\[
		\Gamma(z+1) = \int_0^\infty t^z e^{-t} \, \dd{t}.
	\]
	Aplicamos integración por partes con:
	\[
		u = t^z \Rightarrow du = z t^{z-1} \dd{t}, \quad dv = e^{-t} \dd{t} \Rightarrow v = -e^{-t}.
	\]
	Entonces:
	\[
		\Gamma(z+1) = \left[ -t^z e^{-t} \right]_0^\infty + \int_0^\infty z t^{z-1} e^{-t} \, \dd{t}.
	\]
	El término evaluado en los extremos es cero, ya que:
	\[
		\lim_{t \to \infty} t^z e^{-t} = 0 \quad \text{y} \quad \lim_{t \to 0^+} t^z e^{-t} = 0 \quad \text{para } \operatorname{Re}(z) > 0.
	\]
	Por tanto,
	\[
		\Gamma(z+1) = z \int_0^\infty t^{z-1} e^{-t} \, \dd{t} = z \, \Gamma(z), \qquad   \Re(z)>-1  \text{ y } \Re(z)\neq -1,-2,-3, \cdots \qedhere
	\]
\end{proof}

\nt{De la recurrencia,
	\begin{align*}
		\Gamma(n+z) & = (n-1+z)(n-2+z) \ldots (1+z) \Gamma(1+z) \\
		            & =(n-1+z)(n-2+z) \ldots (1+z) z!           \\
		            & = (n-1+z)!
	\end{align*}}

\ex{}{	Simplifique la siguiente expresión

	$$
		\frac{\Gamma\left(\frac{5}{2} \right)\Gamma\left(\frac{9 }{7} \right)  }{\Gamma\left(\frac{1}{2} \right)\Gamma\left(\frac{2 }{7} \right) }
	$$

	\resolucion

	Sabemos por la Formula de Reducción $\Gamma(z+1)=z \Gamma(z)	$

	$$
		\Gamma\left( \frac{5}{2}\right) =\Gamma\left(1+ \frac{3}{2}\right) =\frac{3}{2}\Gamma\left(\frac{3}{2}\right) = \frac{3}{2}\Gamma\left(1+\frac{1}{2}\right)=\frac{3}{2} \frac{1}{2} \Gamma\left(\frac{1}{2}\right)
	$$

	$$
		\Gamma\left( \frac{9}{7}\right) =\Gamma\left(1+ \frac{2}{7}\right) =\frac{2}{7}\Gamma\left(\frac{2}{7}\right)
	$$

	Reemplazando en la expresión
	$$
		\frac{\Gamma\left(\frac{5}{2} \right)\Gamma\left(\frac{9 }{7} \right)  }{\Gamma\left(\frac{1}{2} \right)\Gamma\left(\frac{2 }{7} \right) }= \frac{\frac{3}{2} \cdotp \frac{1}{2} \Gamma\left(\frac{1}{2}\right) \cdotp      \frac{2}{7}\Gamma\left(\frac{2}{7}\right) }{\Gamma\left(\frac{1}{2} \right)\Gamma\left(\frac{2 }{7} \right) } =\frac{3}{2} \cdotp \frac{1}{2}\cdotp  \frac{2}{7} =\frac{3}{14}
	$$

	$$
		\therefore \frac{\Gamma\left(\frac{5}{2} \right)\Gamma\left(\frac{9 }{7} \right)  }{\Gamma\left(\frac{1}{2} \right)\Gamma\left(\frac{2 }{7} \right) }=\frac{3}{14}
	$$
}
\ex{Encuentre el valor de $\Gamma(\frac{5}{2})$ si $\Gamma(\frac{1}{2}) = \sqrt{\pi}$}{\label{ex:ejmplo_01_01}

	\begin{align*}
		\Gamma\qty(\frac{5}{2}) & = \Gamma\qty(1+\frac{3}{2})  =  \frac{3}{2} \cdot \Gamma\qty(\frac{3}{2}) =  \frac{3}{2} \cdot\pqty{ \Gamma\qty(1+\frac{1}{2}) } = \\
		                        & =  \frac{3}{2} \cdot\pqty{\frac{1}{2}\cdot \Gamma\qty(\frac{1}{2}) } = \frac{3}{4} \cdot \Gamma\qty(\frac{1}{2})                   \\
		                        & = \frac{3}{4} \sqrt{\pi}
	\end{align*}
}

\ex{}{	Calcular la siguiente integral, si $\Gamma\left( \frac{1}{2} \right)=\sqrt{\pi} $

	$$
		\int_{0}^{\infty} t^{7/2} e^{-t}\dd{t}
	$$

	\resolucion

	$$
		\int_{0}^{\infty} t^{7/2} e^{-t}\dd{t}=\int_{0}^{\infty} t^{5/2-1} e^{-t}\dd{t}=\Gamma \left( \frac{5}{2} \right)
	$$

	Usamos la Formula de Reducción $\Gamma(z+1)=z \Gamma(z)	$, usamos el resultado del ejemplo \ref{ex:ejmplo_01_01},
	$$
		\therefore \int_{0}^{\infty} t^{7/2} e^{-t}\dd{t}=\frac{3}{4}\sqrt{\pi}
	$$
}
\ex{}{	Calcular el siguiente límite para todo $a>1$

	$$
		\lim\limits_{n \to \infty} \frac{\Gamma(n!+1) \left( \sqrt[\Gamma(n!+1)]{a}-1\right) }{\Gamma(n-1)}
	$$

	\resolucion

	$$
		\lim\limits_{n \to \infty} \frac{\Gamma(n!+1) \left( \sqrt[\Gamma(n!)]{a}-1\right) }{\Gamma(n-1)}=	\lim\limits_{n \to \infty} \frac{\Gamma(n!+1) \left( \sqrt[\Gamma(n!)]{a}-1\right) }{n!}
	$$
	$$
		\lim\limits_{n \to \infty} \frac{n!\Gamma(n!) \left( \sqrt[\Gamma(n!)]{a}-1\right) }{n!}=\lim\limits_{n \to \infty}  \Gamma(n!) \left( \sqrt[\Gamma(n!)]{a}-1\right)
	$$

	Hacemos el siguiente cambio de variable $x=\frac{1}{\Gamma(n!)}  $

	$$
		\lim\limits_{x \to 0 ^+} \frac{a^x-1}{x}=\ln(a)
	$$
	$$
		\therefore  \lim\limits_{n \to \infty} \frac{\Gamma(n!+1) \left( \sqrt[\Gamma(n!+1)]{a}-1\right) }{\Gamma(n-1)} = \ln(a)
	$$
}

\mprop{Producto de Weierstrass}{
	La función gamma admite la representación infinita:

	\[
		\frac{1}{\Gamma(z)} = z e^{\gamma z} \prod_{n=1}^\infty \left(1 + \frac{z}{n} \right) e^{-z/n}, \quad z \in \mathbb{C} \setminus \{0, -1, -2, \dots\}
	\]
	Es decir,
	$$
		\Gamma(z)=\dfrac{e^{-\gamma z}}{z} \prod_{n=1}^{\infty} \left(1+\dfrac{z}{n} \right)^{-1} e^{z/n}
	$$


	donde la constante \( \gamma \), llamada \textbf{constante de Euler–Mascheroni}, se define como:

	\[
		\gamma := \lim_{m \to \infty} \left( \sum_{k=1}^m \frac{1}{k} - \ln m \right) \approx 0.5772\ 1566\ 4900\ldots
	\]
}

\mprop{Fórmula de reflexión de Euler}{
	Para todo \( z \notin \mathbb{Z} \), se cumple:
	\[
		\Gamma(z) \Gamma(1 - z) = \frac{\pi}{\sin(\pi z)}.
	\]}

\begin{proof}
	Utilizamos la fórmula del producto de Weierstrass para la función Gamma:
	\[
		\frac{1}{\Gamma(z)} = z e^{\gamma z} \prod_{n=1}^\infty \left(1 + \frac{z}{n} \right) e^{-z/n},
	\]

	Entonces:
	\[
		\Gamma(z) = \frac{1}{z} e^{-\gamma z} \prod_{n=1}^\infty \left(1 + \frac{z}{n} \right)^{-1} e^{z/n}.
	\]

	Análogamente:
	\[
		\Gamma(1 - z) = \frac{1}{1 - z} e^{-\gamma(1 - z)} \prod_{n=1}^\infty \left(1 + \frac{1 - z}{n} \right)^{-1} e^{(1 - z)/n}.
	\]

	Además \( \pi / \sin(\pi z) \), es igual a:
	\[
		\frac{\pi}{\sin(\pi z)} = \frac{1}{z} \prod_{n=1}^\infty \left(1 - \frac{z^2}{n^2} \right)^{-1}.
	\]

	\[
		\Rightarrow \Gamma(z)\Gamma(1 - z) = \frac{\pi}{\sin(\pi z)}.
		\qedhere
	\]
\end{proof}

\ex{$\Gamma(\frac{1}{2}) = \sqrt{\pi}$}{
	Haciendo $z = \frac{1}{2}$ en la reflexión de Euler
	\begin{align*}
		\Gamma\qty(\frac{1}{2}) \Gamma\qty(1-\frac{1}{2}) & = \frac{\pi}{\sin(\pi \cdot \frac{1}{2})} \\
		\pqty{\Gamma\qty(\frac{1}{2})}^2                  & = \pi                                     \\
		\Gamma\qty(\frac{1}{2})                           & = \sqrt{\pi}
	\end{align*}
}

\ex{ Calcule el límite usando la formula de Reflexión de Euler-Función Gamma}{

	$$
		\lim_{x \to  0 }\frac{\sin(x)}{x}
	$$
	\resolucion

	Usamos Reflexión de Euler

	$$
		\lim_{x \to  0 } \frac{1}{x} \cdot \frac{\pi}{\Gamma\left( \ \frac{x}{\pi}\right)  \Gamma\left( 1-\frac{ x}{\pi} \right)  }
	$$

	$$
		\lim_{x \to  0 } \frac{1}{\dfrac{x}{\pi}} \cdot \frac{1}{\Gamma\left( \frac{x}{\pi}\right)  \Gamma\left( 1-\frac{ x}{\pi} \right)  }
	$$

	Hacemos un cambio de variable $x/\pi=z \Rightarrow  z \to 0$

	$$
		\lim_{z \to  0 } \dfrac{1}{z} \cdot \frac{1}{\Gamma\left( z\right)  \Gamma\left( 1-z \right)  }
	$$

	Sabemos que $\Gamma(z+1)=z\Gamma(z)$

	$$
		\lim_{z \to  0 } \cdot \frac{1}{\Gamma\left( z+1\right)  \Gamma\left( 1-z \right)  } =\frac{1}{\Gamma(1+0)\Gamma(1-0)}=1
	$$

	$$
		\therefore \lim_{x \to  0 }\frac{\sin(x)}{x}=1
	$$}
\section{ Otras Propiedades Fundamentales}

\begin{itemize}

	\item \textbf{Fórmula de Stirling (asimptótica):}
	      \[
		      \Gamma(z) \sim \sqrt{2\pi} z^{z - 1/2} e^{-z} \left( 1 + \frac{1}{12z} + \frac{1}{288z^2} - \cdots \right)
	      \]
	\item \textbf{Fórmula de duplicación de Legendre:}
	      \[
		      \Gamma(2z) = \frac{\Gamma(z)\Gamma\left(z + \tfrac{1}{2} \right)}{2^{1 - 2z} \sqrt{\pi}}
	      \]

	      $$
		      2^{2z-1}\Gamma(z)\Gamma\left( z+\dfrac{1}{2}\right) =\sqrt{\pi}\Gamma(2z)
	      $$

	      $$
		      \Gamma\left( n+\dfrac{1}{2}\right) =\dfrac{(2n)!}{4^n n!}\sqrt{\pi} \hspace{1cm} n=1,2,3,\ldots
	      $$

	\item \textbf{Fórmula de triplicación:}

	      \[
		      \Gamma(3z) = \frac{3^{3z - \frac{1}{2}}}{2\pi} \, \Gamma(z) \, \Gamma\left(z + \frac{1}{3} \right) \, \Gamma\left(z + \frac{2}{3} \right)
	      \]

	\item \textbf{Fórmula de multiplicación de Gauss:}

	      Para \( n \in \mathbb{N} \), se tiene:

	      \[
		      \Gamma(nz) = (2\pi)^{\frac{1-n}{2}} \, n^{nz - \frac{1}{2}} \prod_{k=0}^{n-1} \Gamma\left(z + \frac{k}{n} \right)
	      \]
	      $$
		      (2\pi)^{(n-1)/2}  n^{1/2-nz}\Gamma(nz)=\prod_{k=0}^{n-1}\Gamma\left( z+\dfrac{k}{n}\right)
	      $$

	      $$
		      {\displaystyle \Gamma (z)\;\Gamma \left(z+{\frac {1}{k}}\right)\;\Gamma \left(z+{\frac {2}{k}}\right)\cdots \Gamma \left(z+{\frac {k-1}{k}}\right)=(2\pi )^{\frac {k-1}{2}}\;k^{1/2-kz}\;\Gamma (kz)\,\!}
	      $$
	\item \textbf{Representación de Gauss }
	      \[
		      \Gamma(z) = \lim_{n \to \infty} \frac{n! \, n^z}{z (z+1) (z+2) \cdots (z+n)} = \lim_{n \to \infty} \frac{n! \, n^z}{\prod_{k=0}^{n} (z + k)}
		      \quad \text{para } z \notin \{0, -1, -2, \dots\}
	      \]
	\item \textbf{Formula de Knar's  }

	      $$
		      \Gamma(1+z)=2^{2z}\prod_{k=1}^{\infty} \left(\pi^{-1/2} \Gamma\left( \dfrac{1}{2}+2^{-k}z \right)  \right)
	      $$

	\item \textbf{Euler's Formula:}

	      $$
		      \Gamma(z)=\lim\limits_{n \to \infty} \dfrac{n^z}{z}\prod_{k=1}^{n} \dfrac{k}{k+z}=\dfrac{1}{z}\prod_{k=}^{\infty}\dfrac{\left(1+\frac{1}{k} \right)^z }{1+\frac{z}{k}}
	      $$

	\item \textbf{Coeficiente binomial generalizado:}

	      \[
		      \binom{z}{w} = \frac{\Gamma(z+1)}{\Gamma(w+1)\Gamma(z-w+1)}
		      \quad \text{para } z, w \in \mathbb{C} \setminus \mathbb{Z}_{< -1}
	      \]

	\item \textbf{Propiedades en el plano complejo:}

	      \begin{align*}
		      \Gamma(\overline{z})     & = \overline{\Gamma(z)}                                        \\
		      \ln \Gamma(\overline{z}) & = \overline{\ln \Gamma(z)}                                    \\
		      |\Gamma(x + i y)|        & \leq |\Gamma(x)| \quad \text{para } x > 0,\, y \in \mathbb{R} \\
		      \arg \Gamma(z+1)         & = \arg \Gamma(z) + \arg z
	      \end{align*}

	\item \textbf{Serie de potencias de \( \ln \Gamma(1+z) \):}

	      Para \( |z| < 1 \):

	      \[
		      \ln \Gamma(1 + z) = -\gamma z + \sum_{n=2}^\infty \frac{\zeta(n)}{n} (-z)^n
	      \]

	      donde \( \zeta(n) \) es la función zeta de Riemann.

	\item \textbf{Expansión en serie de \( \dfrac{1}{\Gamma(z)} \):}

	      Hay una expansión entera (ver proposición de Weierstrass),  también existe una expansión de tipo:

	      \[
		      \frac{1}{\Gamma(z)} = \sum_{n=1}^\infty c_n z^n, \quad \text{con } c_n \in \mathbb{R}
		      \quad \text{(convergente en } \mathbb{C})
	      \]

	\item \textbf{Aproximación de Stirling's}


	      Para \( |z| \to \infty \), \( |\arg z| < \pi \):

	      \[
		      \Gamma(z) \sim \sqrt{2\pi} \, z^{z - \frac{1}{2}} e^{-z} \left( 1 + \frac{1}{12z} + \frac{1}{288z^2} - \frac{139}{51840z^3} + \cdots \right)
	      \]

	      La serie asintomática para la función Gamma
	      $$
		      \Gamma(n+1) \sim  e^{-n} n^n \sqrt{2 \pi n} \left(1+\dfrac{1}{12n}+\dfrac{1}{288n^2} - \dfrac{139}{51840z^3}-\dfrac{571}{2488320z^4}+\ldots\right)
	      $$

	      Para $n$ suficientemente grande,
	      $$
		      \ln(n!)\sim n\ln(n)-n
	      $$

	      En los problemas vamos a usar con más recuerrencia

	      $$
		      n!\sim \sqrt{2\pi n}\left(\dfrac{n}{e} \right)^n
	      $$

	      \ex{ Calcule el siguiente límite}{\label{ejem1}

		      $$
			      \lim\limits_{n \to \infty} \dfrac{\sqrt[n]{n!}}{n}
		      $$

		      \resolucion

		      Usamos la Aproximación de Stirling $  n!\approx n^n e^{-n}\sqrt{2\pi n }$

		      $$
			      \lim_{n\to \infty}  \dfrac{\sqrt[n]{n^n e^{-n}\sqrt{2\pi n} }}{n}=	\lim_{n\to \infty}  \dfrac{ n e^{-1}\sqrt[2n]{2\pi n }}{n}
		      $$

		      Se sabe que $\lim_{n\to\infty} \sqrt[2n]{2\pi n}=1$

		      $$
			      \therefore 	\lim\limits_{n \to \infty} \dfrac{\sqrt[n]{n!}}{n}=\dfrac{1}{e}
		      $$
	      }

	      \ex{Encuentre el valor del límite si  $k=1,2,3,\ldots$ }{
		      $$
			      \lim\limits_{n \to \infty} \dfrac{\ln(n^n)}{\ln((kn)!)}
		      $$
		      \resolucion
		      Usamos la Aproximación de Stirling $  n!\sim n^n e^{-n}\sqrt{2\pi n }$

		      $$
			      \lim\limits_{n \to \infty} \dfrac{\ln(n^n)}{\ln((kn)^{(kn)} e^{-kn}\sqrt{2\pi kn })}
		      $$

		      Sabemos que $\ln(ab)=\ln(a)+\ln(b) ; a,b >0$

		      $$
			      \lim\limits_{n \to \infty} \dfrac{n\ln(n)}{kn\ln(kn/e) +\ln(\sqrt{2\pi kn })}
		      $$

		      Dividimos entre $n$ al numerador como denominador

		      $$
			      \lim\limits_{n \to \infty} \dfrac{\ln(n)}{k\ln(kn/e) + \ln(\sqrt[2n]{2\pi kn })}
		      $$

		      Es facil comprobar que $\lim_{n\to\infty} \ln( \sqrt[2n]{2\pi k n} ) =0$, además $k/e$ es una constante $c$

		      $$
			      \lim\limits_{n \to \infty} \dfrac{\ln(n)}{k\ln(cn)}=\dfrac{1}{k}
		      $$

		      $$
			      \therefore \lim\limits_{n \to \infty} \dfrac{\ln(n^n)}{\ln((kn)!)}=\dfrac{1}{k}
		      $$
	      }


	      \ex{Encuentre el valor del límite si  $k=1,2,3,\ldots$}{
		      $$
			      \lim\limits_{n \to \infty} \dfrac{\ln(n^n)}{\ln((kn)!)}
		      $$

		      \resolucion

		      {\bf Método 1}

		      Usamos la Aproximación de Stirling $  n!\sim n^n e^{-n}\sqrt{2\pi n }$

		      $$
			      \lim\limits_{n \to \infty} \dfrac{\ln(n^n)}{\ln((kn)^{(kn)} e^{-kn}\sqrt{2\pi kn })}
		      $$

		      Sabemos que $\ln(ab)=\ln(a)+\ln(b) ; a,b >0$

		      $$
			      \lim\limits_{n \to \infty} \dfrac{n\ln(n)}{kn\ln(kn/e) +\ln(\sqrt{2\pi kn })}
		      $$

		      Dividimos entre $n$ al numerador como denominador

		      $$
			      \lim\limits_{n \to \infty} \dfrac{\ln(n)}{k\ln(kn/e) + \ln(\sqrt[2n]{2\pi kn })}
		      $$

		      Es facil comprobar que $\lim_{n\to\infty} \ln( \sqrt[2n]{2\pi k n} ) =0$, además $k/e$ es una constante $c$

		      $$
			      \lim\limits_{n \to \infty} \dfrac{\ln(n)}{k\ln(cn)}=\dfrac{1}{k}
		      $$

		      $$
			      \therefore \lim\limits_{n \to \infty} \dfrac{\ln(n^n)}{\ln((kn)!)}=\dfrac{1}{k}
		      $$

	      }


	      \ex{}{ \label{ejem2}   Encuentre el valor del límite para todo $k=1,2,3,\ldots $

		      $$
			      \lim_{n \to \infty} \left( \frac{(kn)!}{(k!n^k )^n  }\right)^{1/kn}
		      $$
		      \resolucion

		      Aproximación de Stirling $  t!\sim t^t e^{-t}\sqrt{2\pi t}$ si $ t \to \infty $

		      $$
			      \lim_{n \to \infty} \left( \frac{ (kn)^{kn} e^{-kn} \ \sqrt{2\pi kn} }{  (k! n^k)^n  }\right)^{1/kn}
		      $$

		      $$
			      \lim_{n \to \infty}  \frac{ (kn) e^{-1} \sqrt[2kn]{2\pi kn} }{  \sqrt[k]{k!} n }
		      $$

		      En los ejemplos anteriores, vimos que  $\lim_{n\to\infty}  \sqrt[2nk]{2\pi k n} =1$
		      $$
			      \lim_{n \to \infty}  \dfrac{ k \sqrt[2kn]{2\pi kn} }{ e \sqrt[k]{k!}  } = \frac{ k}{ e \sqrt[k]{k!}  }
		      $$


		      $$
			      \therefore \lim_{n \to \infty} \left( \frac{(kn)!}{(k!n^k )^n  }\right)^{1/kn} =\frac{ k}{ e \sqrt[k]{k!}  }
		      $$

	      }

	      \ex{Calcule el siguiene límite $ k, n >0$}{
	      $$
		      \lim_{x \to \infty} \left(\frac{(x+k)^{x^{n+1}} \sqrt{x^{2x^n-x^{n-1}} }    }{x^{x^{n+1}} \pqty{\Gamma(x)}^{x^{n-1}} } \right) ^{x^{-n}}
	      $$

	      \resolucion

	      Sea $L$ el límite a calcular, reconocemos $(ab)^n=a^n \cdot b^n$
	      $$
		      L=  \lim_{x \to \infty}\left(\dfrac{(x+k)^{x^{n+1}}     }{x^{x^{n+1}} } \right) ^{x^{-n}}\cdot
		      \left(\dfrac{ \sqrt{x^{2x^n-x^{n-1}} }    }{ \pqty{\Gamma(x)}^{x^{n-1}} } \right) ^{x^{-n}}
	      $$

	      Tambien vemos $a^n \cdot a^m=a^{n+m}$

	      $$
		      L=  \lim_{x \to \infty}\dfrac{(x+k)^{x^{n+1-n}}     }{x^{x^{n+1-n}} } \cdot
		      \lim_{x \to \infty} \left(\dfrac{ \sqrt{x^{2x^{n-n+1}-x^{n-1-n+1}} }    }{ \pqty{\Gamma(x)}^{x^{n-1-n+1}} } \right)^{x^{-1}}
	      $$

	      $$
		      L= \lim_{x \to \infty}\dfrac{(x+k)^x     }{x^x } \cdot
		      \lim_{x \to \infty} \left(\dfrac{ \sqrt{x^{2x-1} }    }{ \Gamma(x) } \right)^{x^{-1}}
	      $$

	      Relacionamos la función Gamma con el factorial $\Gamma(x)=(x-1)!$ \ \ y \ \     $ x(x-1)!=x! $

	      $$
		      L=  \lim_{x \to \infty} \left( \dfrac{x+k    }{x }\right)^x  \cdot
		      \lim_{x \to \infty} \left(\dfrac{x^{x-1/2} }{ (x-1)! } \right)^{x^{-1}}
		      =
		      e^k \cdot
		      \lim_{x \to \infty} \left(\dfrac{x^{x+1/2} }{ x! } \right)^{x^{-1}}
	      $$

	      Aproximación de Stirling $  n!\sim n^n e^{-n}\sqrt{2\pi n }$

	      $$
		      L=   e^k\lim_{x \to \infty} \left(\dfrac{x^{x+1/2} }{\sqrt{2\pi}\  x^{x+1/2} \ e^{-x}} \right)^{x^{-1}}
		      =
		      e^k\lim_{x \to \infty} \left(\dfrac{e^x }{\sqrt{2\pi}  } \right)^{x^{-1}}
	      $$

	      Facilmente se comprueba que  $\lim_{x\to\infty} \sqrt[2x]{2\pi }=1$, entonces
	      $$
		      L= e^k\lim_{x \to \infty} \dfrac{e}{\sqrt[2x]{2\pi}  } =e^k\cdot e=e^{k+1}
	      $$
	      Finalmente, tenemos el valor de $L$,
	      $$
		      \therefore  \lim_{x \to \infty} \left(\dfrac{(x+k)^{x^{n+1}} \sqrt{x^{2x^n-x^{n-1}} }    }{x^{x^{n+1}} \Gamma^{x^{n-1}}(x) } \right) ^{x^{-n}}=e^{k+1}
	      $$
	      }

	\item \textbf{Desarrollo asintótico de \( \ln \Gamma(z) \):}

	      \[
		      \ln \Gamma(z) \sim z \ln z - z + \frac{1}{2} \ln(2\pi) + \sum_{n=1}^\infty \frac{B_{2n}}{2n(2n-1) z^{2n-1}}
		      \quad \text{para } |z| \to \infty,\, |\arg z| < \pi
	      \]

	      donde \( B_{2n} \) son los números de Bernoulli.

	\item \textbf{Expansión de Laurent}

	      $$
		      \Gamma(z)=\frac{1}{z}-\gamma +\frac{(\gamma^2+\zeta(2))z }{2}+\mathcal{O}(z^2)
	      $$

	      En los siguientes capitulos estudiaremos las siguientes constantes:

	      La constante de Euler Mascheroni $\gamma \approx 0.577 \ldots $

	      La función Zeta de Riemman de dos $\zeta(2)=\frac{\pi^2}{6}$

	      \ex{Calcular el siguiente límite}{
		      $$
			      \lim\limits_{t \to 0} \frac{\Gamma(t+1)+\gamma t-1}{t^2}
		      $$

		      donde $\gamma$ es la constante de Euler-Mascheroni

		      \resolucion

		      Por la expansión de Laurent
		      $$
			      \Gamma(t+1)=1-\gamma t + \frac{(\gamma^2+\zeta(2))t^2 }{2}+\mathcal{O}(t^3)
		      $$

		      $$
			      \Gamma(t+1)+\gamma t-1=  \frac{(\gamma^2+\zeta(2))t^2 }{2}+\mathcal{O}(t^3)
		      $$

		      $$
			      \frac{\Gamma(t+1)+\gamma t-1}{t^2}= \frac{(\gamma^2+\zeta(2))}{2}+\mathcal{O}(t)
		      $$

		      $$
			      \lim\limits_{t \to 0 }	\frac{\Gamma(t+1)+\gamma t-1}{t^2}=\lim\limits_{t \to 0 } \left(   \frac{(\gamma^2+\zeta(2))}{2}+\mathcal{O}(t)\right)
		      $$


		      $$
			      \lim\limits_{t \to 0 }	\frac{\Gamma(t+1)+\gamma t-1}{t^2}=   \frac{(\gamma^2+\zeta(2))}{2}+\cancelto{0}{\lim\limits_{t \to 0 } \mathcal{O}(t)}
		      $$

		      La función Zeta de Riemman de dos es,  $\zeta(2)=\frac{\pi^2}{6}$

		      $$
			      \therefore \lim\limits_{t \to 0 }	\frac{\Gamma(t+1)+\gamma t-1}{t^2}=\frac{\gamma^2}{2}+\frac{\pi^2}{12}
		      $$
	      }
\end{itemize}

\section{Ejercicios }

\subsection{Ejercicios Resueltos}
\begin{enumerate}
	\item Evalue:
	      \[
		      \Gamma\left(\frac{1+i}{2}\right) \Gamma\left(\frac{1-i}{2}\right)
	      \]
	      Solución:
	      \begin{align*}
		      \Gamma\left(\frac{1+i}{2}\right) \Gamma\left(\frac{1-i}{2}\right)
		       & = \Gamma\left(\frac{1+i}{2}\right) \Gamma\left(1 - \frac{1+i}{2}\right)                                                         \\
		       & = \frac{\pi}{\sin\left(\pi \cdot \pqty{\frac{1+i}{2} } \right)} = \frac{\pi}{\sin\left(\frac{\pi}{2} + \frac{\pi i}{2} \right)} \\
		       & = \frac{\pi}{\cos(\frac{\pi i}{2} )} =\frac{\pi}{\cosh(\frac{\pi }{2} )}                                                        \\
		       & =   \pi  \sech(\frac{\pi}{2})
	      \end{align*}
	\item  Cacule,
	      $$
		      \int_a^{a+1} \ln(\Gamma(x)) \dd{x}
	      $$
	      Sol:

	      Sea $f(a)$ la integral,
	      $$
		      f(a)=\int_a^{a+1} \ln(\Gamma(x)) \dd{x}
	      $$
	      diferenciando ambos lados
	      $$
		      f'(a) = \ln(\Gamma(a+1)) -\ln(\Gamma(a)) = \ln(a)
	      $$
	      Integrando respecto a $a$ en ambos lados,
	      $$
		      f(a)= a\ln(a)-a+C
	      $$
	      si $a \to 0^{+}$, tenemos que
	      $$
		      C = \int_0^1 \ln(\Gamma(x)) \dd{x} = \ln(\sqrt{2\pi})
	      $$
	      Finalmente,
	      $$
		      \int_a^{a+1} \ln(\Gamma(x)) \dd{x} = a\ln(a)-a+\ln(\sqrt{2\pi}) = \ln(\sqrt{2\pi} \pqty{\frac{a}{e}}^a)
	      $$
	\item 	$$
		      \lim_{k \to \infty}  \ \lim_{n \to \infty} \left( \frac{(kn)!}{(k!n^k )^n  }\right)^{1/kn}
	      $$

	      \resolucion

	      Por el Ejemplo \ref{ejem2}

	      $$
		      \lim_{k \to \infty}\left[   \ \lim_{n \to \infty} \left( \dfrac{(kn)!}{(k!n^k )^n  }\right)^{1/kn}\right]  = \lim_{k \to \infty} \left[ \dfrac{ k}{ e \sqrt[k]{k!}  } \right]
	      $$

	      \noindent

	      Por el Ejemplo \ref{ejem1}

	      $$
		      \lim_{k \to \infty} \left[ \dfrac{ k}{ e \sqrt[k]{k!}  } \right] =\dfrac{1}{e \left( \dfrac{1}{e} \right) }=1
	      $$

	      $$
		      \therefore \lim_{k \to \infty}  \ \lim_{n \to \infty} \left( \frac{(kn)!}{(k!n^k )^n  }\right)^{1/kn} =1
	      $$
	\item
\end{enumerate}


\subsection{Gamma Incompleta}

\dfn{Función Gamma Incompleta}{
	$$
		\Gamma(z,x)=\int_x^{\infty} t^{z-1 } e^{-t} \dd{t}  \hspace{1cm} \Re(z)>0
	$$
}

\dfn{Función gamma Incompleta- Función Gamma Inferior}{
	$$
		\gamma(z,x)=\int_0^{x} t^{z-1 } e^{-t} \dd{t}  \hspace{1cm} \Re(z)>0
	$$
}

\dfn{Simbolo de Kramp}{
	$$
		c^{a/b}=c(c+b)(c+2b)\ldots\left[ c+(a-1)b \right] =\dfrac{b^a  \Gamma\left(a+ \dfrac{c}{b}\right)}{\Gamma\left( \dfrac{c}{b}\right) }
	$$ }

\thm{}{

	Con la representaciónde la función Gamma, podemos extender  a 	$x > -1$.

	$$
		\bigg\rvert  \int_0^{\infty} e^{-t}t^x\dd{t} \bigg\rvert  \leq  \int_{\epsilon}^{\infty} e^{-t}t^x\dd{t} <\infty
	$$

	Cuando cercamos a cero $x=-z$ se tiene

	$$
		\bigg\rvert  \int_0^{\epsilon} \dfrac{e^{-t}}{t^z }  \dd{t} \bigg\rvert  \sim   \int_0^{\epsilon} \dfrac{1}{t^z}\dd{t} <\infty
	$$
}

\subsubsection{Relación entre la Función Gamma Superior e Inferior}

\[ \Gamma(z)=\gamma(z,x)+\Gamma(z,x) \left\{ \begin{array}{rcl}
		 & \Gamma(z+1,x)=z\Gamma(z,x)+x^z e^{-x} \\
		 &                                       \\
		 & \gamma(z+1,x)=z\gamma(z,x)-x^z e^{-x} \\
	\end{array}
	\right. \]

$$
	\Gamma(z)=\lim_{x\to\infty} \gamma(z,x)
$$


\subsection{Ejercicios Propuestos}

\begin{enumerate}
	\item Pruebe,
	$$
	\int_0^1 \ln(\Gamma(x)) \dd{x} = \ln(\sqrt{2\pi})
	$$
	
	\item Evaluar la siguiente integral,
	\[
	\int_0^\infty x^n e^{-\lambda x} \, dx
	\quad \textbf{Respuesta: } \frac{\Gamma(n+1)}{\lambda^{n+1}}
	\]
	
	\item Evaluar:
	\[
	\int_0^1 (-\ln t)^s \, \dd{t}
	\quad \textbf{Respuesta: } \Gamma(s+1)
	\]
	
	\item Demostrar que:
	\[
	\int_0^\infty e^{-x^2} \, dx = \frac{\sqrt{\pi}}{2}
	\quad \text{usando } \Gamma\left(\frac{1}{2}\right)
	\]
	
	\item Mostrar que:
	\[
	\int_0^\infty \frac{t^{z-1}}{1 + t} \, \dd{t} = \pi / \sin(\pi z)
	\quad \text{para } 0 < \operatorname{Re}(z) < 1
	\]
	\item
	Encuentre el valor del límite si $a \in \mathbb{Z^{+}}$
	
	$$
	\lim_{k \to \infty}    \left(\frac{e^{(k!)^{-a} +(k!)^k}-k^{(k^2)!}}{\left( (ak)!\right) ^{ak}} \right)
	$$
	\item  Evalúa,
	$$
	\int_0^{\infty} \sqrt[4]{x} e^{-\sqrt{x}} \dd{x}
	$$
	
	\item  Calcule,
	$$
	\int_0^1 x^5 \sqrt[3]{\ln(\frac{1}{x})} \dd{x}
	$$
	
	\item Pruebe, 
	$$
	\int_0^{\infty } e^{-\alpha t^2} \cos(\beta t) \dd{t} = \frac{\sqrt{\pi}}{2\sqrt{\alpha}} e^{-\frac{-\beta^2}{4\alpha}}
	$$

	\item Si $a_n:=n!^n$, pruebe que
	\item 
	$$
	\lim\limits_{n \to \infty } (a_n-a_{n-1}) = \frac{1}{e}
	$$
	\item Si $a>0$, pruebe que 
	\item 
	$$
	\lim\limits_{x \to 1^{-}} \pqty{\sqrt[a]{1-x} \cdot \sum_{n=0}^{\infty} x^{n^a}} = \Gamma\qty(1+\frac{1}{a})
	$$
	\item Si $k $ es par, evalúe
	\item 
	$$
	\int_0^1 \frac{\Gamma\pqty{\frac{1}{3}} + \sqrt[3]{-\ln(x)} \cdot x^n \cdot (\ln(x))^k}{\sqrt[3]{-\ln(x)}} \dd{x}
	$$
	\item Demuestre que,
	\item 
	$$
	\int_0^{\infty} \frac{x^{m-1}}{1+x^n} = \frac{\pi}{n \sin(\frac{m}{n} \pi)}
	$$
	Sabiendo que 

	$$
	\int_0^{\infty} \frac{x^{p-1}}{1+x} \dd{x} = \frac{\pi}{\sin(p \pi)} 
	$$
	
\end{enumerate}
