\documentclass{report}
\usepackage[spanish]{babel}
%%%%%%%%%%%%%%%%%%%%%%%%%%%%%%%%%
% PACKAGE IMPORTS
%%%%%%%%%%%%%%%%%%%%%%%%%%%%%%%%%


\usepackage[tmargin=2cm,rmargin=1in,lmargin=1in,margin=0.85in,bmargin=2cm,footskip=.2in]{geometry}
\usepackage{amsmath,amsfonts,amsthm,amssymb,mathtools}
\usepackage[varbb]{newpxmath}
\usepackage{xfrac}
\usepackage[makeroom]{cancel}
\usepackage{mathtools}
\usepackage{bookmark}
\usepackage{enumitem}
\usepackage{hyperref,theoremref}
\hypersetup{
	pdftitle={Assignment},
	colorlinks=true, linkcolor=doc!90,
	bookmarksnumbered=true,
	bookmarksopen=true
}
\usepackage[most,many,breakable]{tcolorbox}
\usepackage{xcolor}
\usepackage{varwidth}
\usepackage{varwidth}
\usepackage{etoolbox}
%\usepackage{authblk}
\usepackage{nameref}
\usepackage{multicol,array}
\usepackage{tikz-cd}
\usepackage[ruled,vlined,linesnumbered]{algorithm2e}
\usepackage{comment} % enables the use of multi-line comments (\ifx \fi) 
\usepackage{import}
\usepackage{xifthen}
\usepackage{pdfpages}
\usepackage{transparent}

\newcommand\mycommfont[1]{\footnotesize\ttfamily\textcolor{blue}{#1}}
\SetCommentSty{mycommfont}
\newcommand{\incfig}[1]{%
    \def\svgwidth{\columnwidth}
    \import{./figures/}{#1.pdf_tex}
}

\usepackage{tikzsymbols}
\renewcommand\qedsymbol{$\Laughey$}


%\usepackage{import}
%\usepackage{xifthen}
%\usepackage{pdfpages}
%\usepackage{transparent}


%%%%%%%%%%%%%%%%%%%%%%%%%%%%%%
% SELF MADE COLORS
%%%%%%%%%%%%%%%%%%%%%%%%%%%%%%



\definecolor{myg}{RGB}{56, 140, 70}
\definecolor{myb}{RGB}{45, 111, 177}
\definecolor{myr}{RGB}{199, 68, 64}
\definecolor{mytheorembg}{HTML}{F2F2F9}
\definecolor{mytheoremfr}{HTML}{00007B}
\definecolor{mylenmabg}{HTML}{FFFAF8}
\definecolor{mylenmafr}{HTML}{983b0f}
\definecolor{mypropbg}{HTML}{f2fbfc}
\definecolor{mypropfr}{HTML}{191971}
\definecolor{myexamplebg}{HTML}{F2FBF8}
\definecolor{myexamplefr}{HTML}{88D6D1}
\definecolor{myexampleti}{HTML}{2A7F7F}
\definecolor{mydefinitbg}{HTML}{E5E5FF}
\definecolor{mydefinitfr}{HTML}{3F3FA3}
\definecolor{notesgreen}{RGB}{0,162,0}
\definecolor{myp}{RGB}{197, 92, 212}
\definecolor{mygr}{HTML}{2C3338}
\definecolor{myred}{RGB}{127,0,0}
\definecolor{myyellow}{RGB}{169,121,69}
\definecolor{myexercisebg}{HTML}{F2FBF8}
\definecolor{myexercisefg}{HTML}{88D6D1}


%%%%%%%%%%%%%%%%%%%%%%%%%%%%
% TCOLORBOX SETUPS
%%%%%%%%%%%%%%%%%%%%%%%%%%%%

\setlength{\parindent}{1cm}
%================================
% THEOREM BOX
%================================

\tcbuselibrary{theorems,skins,hooks}
\newtcbtheorem[number within=section]{Theorem}{Theorem}
{%
	enhanced,
	breakable,
	colback = mytheorembg,
	frame hidden,
	boxrule = 0sp,
	borderline west = {2pt}{0pt}{mytheoremfr},
	sharp corners,
	detach title,
	before upper = \tcbtitle\par\smallskip,
	coltitle = mytheoremfr,
	fonttitle = \bfseries\sffamily,
	description font = \mdseries,
	separator sign none,
	segmentation style={solid, mytheoremfr},
}
{th}

\tcbuselibrary{theorems,skins,hooks}
\newtcbtheorem[number within=chapter]{theorem}{Theorem}
{%
	enhanced,
	breakable,
	colback = mytheorembg,
	frame hidden,
	boxrule = 0sp,
	borderline west = {2pt}{0pt}{mytheoremfr},
	sharp corners,
	detach title,
	before upper = \tcbtitle\par\smallskip,
	coltitle = mytheoremfr,
	fonttitle = \bfseries\sffamily,
	description font = \mdseries,
	separator sign none,
	segmentation style={solid, mytheoremfr},
}
{th}


\tcbuselibrary{theorems,skins,hooks}
\newtcolorbox{Theoremcon}
{%
	enhanced
	,breakable
	,colback = mytheorembg
	,frame hidden
	,boxrule = 0sp
	,borderline west = {2pt}{0pt}{mytheoremfr}
	,sharp corners
	,description font = \mdseries
	,separator sign none
}

%================================
% Corollery
%================================
\tcbuselibrary{theorems,skins,hooks}
\newtcbtheorem[number within=section]{Corollary}{Corollary}
{%
	enhanced
	,breakable
	,colback = myp!10
	,frame hidden
	,boxrule = 0sp
	,borderline west = {2pt}{0pt}{myp!85!black}
	,sharp corners
	,detach title
	,before upper = \tcbtitle\par\smallskip
	,coltitle = myp!85!black
	,fonttitle = \bfseries\sffamily
	,description font = \mdseries
	,separator sign none
	,segmentation style={solid, myp!85!black}
}
{th}
\tcbuselibrary{theorems,skins,hooks}
\newtcbtheorem[number within=chapter]{corollary}{Corollary}
{%
	enhanced
	,breakable
	,colback = myp!10
	,frame hidden
	,boxrule = 0sp
	,borderline west = {2pt}{0pt}{myp!85!black}
	,sharp corners
	,detach title
	,before upper = \tcbtitle\par\smallskip
	,coltitle = myp!85!black
	,fonttitle = \bfseries\sffamily
	,description font = \mdseries
	,separator sign none
	,segmentation style={solid, myp!85!black}
}
{th}


%================================
% LENMA
%================================

\tcbuselibrary{theorems,skins,hooks}
\newtcbtheorem[number within=section]{Lenma}{Lenma}
{%
	enhanced,
	breakable,
	colback = mylenmabg,
	frame hidden,
	boxrule = 0sp,
	borderline west = {2pt}{0pt}{mylenmafr},
	sharp corners,
	detach title,
	before upper = \tcbtitle\par\smallskip,
	coltitle = mylenmafr,
	fonttitle = \bfseries\sffamily,
	description font = \mdseries,
	separator sign none,
	segmentation style={solid, mylenmafr},
}
{th}

\tcbuselibrary{theorems,skins,hooks}
\newtcbtheorem[number within=chapter]{lenma}{Lenma}
{%
	enhanced,
	breakable,
	colback = mylenmabg,
	frame hidden,
	boxrule = 0sp,
	borderline west = {2pt}{0pt}{mylenmafr},
	sharp corners,
	detach title,
	before upper = \tcbtitle\par\smallskip,
	coltitle = mylenmafr,
	fonttitle = \bfseries\sffamily,
	description font = \mdseries,
	separator sign none,
	segmentation style={solid, mylenmafr},
}
{th}


%================================
% PROPOSITION
%================================

\tcbuselibrary{theorems,skins,hooks}
\newtcbtheorem[number within=section]{Prop}{Proposition}
{%
	enhanced,
	breakable,
	colback = mypropbg,
	frame hidden,
	boxrule = 0sp,
	borderline west = {2pt}{0pt}{mypropfr},
	sharp corners,
	detach title,
	before upper = \tcbtitle\par\smallskip,
	coltitle = mypropfr,
	fonttitle = \bfseries\sffamily,
	description font = \mdseries,
	separator sign none,
	segmentation style={solid, mypropfr},
}
{th}

\tcbuselibrary{theorems,skins,hooks}
\newtcbtheorem[number within=chapter]{prop}{Proposition}
{%
	enhanced,
	breakable,
	colback = mypropbg,
	frame hidden,
	boxrule = 0sp,
	borderline west = {2pt}{0pt}{mypropfr},
	sharp corners,
	detach title,
	before upper = \tcbtitle\par\smallskip,
	coltitle = mypropfr,
	fonttitle = \bfseries\sffamily,
	description font = \mdseries,
	separator sign none,
	segmentation style={solid, mypropfr},
}
{th}


%================================
% CLAIM
%================================

\tcbuselibrary{theorems,skins,hooks}
\newtcbtheorem[number within=section]{claim}{Claim}
{%
	enhanced
	,breakable
	,colback = myg!10
	,frame hidden
	,boxrule = 0sp
	,borderline west = {2pt}{0pt}{myg}
	,sharp corners
	,detach title
	,before upper = \tcbtitle\par\smallskip
	,coltitle = myg!85!black
	,fonttitle = \bfseries\sffamily
	,description font = \mdseries
	,separator sign none
	,segmentation style={solid, myg!85!black}
}
{th}



%================================
% Exercise
%================================

\tcbuselibrary{theorems,skins,hooks}
\newtcbtheorem[number within=section]{Exercise}{Exercise}
{%
	enhanced,
	breakable,
	colback = myexercisebg,
	frame hidden,
	boxrule = 0sp,
	borderline west = {2pt}{0pt}{myexercisefg},
	sharp corners,
	detach title,
	before upper = \tcbtitle\par\smallskip,
	coltitle = myexercisefg,
	fonttitle = \bfseries\sffamily,
	description font = \mdseries,
	separator sign none,
	segmentation style={solid, myexercisefg},
}
{th}

\tcbuselibrary{theorems,skins,hooks}
\newtcbtheorem[number within=chapter]{exercise}{Exercise}
{%
	enhanced,
	breakable,
	colback = myexercisebg,
	frame hidden,
	boxrule = 0sp,
	borderline west = {2pt}{0pt}{myexercisefg},
	sharp corners,
	detach title,
	before upper = \tcbtitle\par\smallskip,
	coltitle = myexercisefg,
	fonttitle = \bfseries\sffamily,
	description font = \mdseries,
	separator sign none,
	segmentation style={solid, myexercisefg},
}
{th}

%================================
% EXAMPLE BOX
%================================

\newtcbtheorem[number within=section]{Example}{Example}
{%
	colback = myexamplebg
	,breakable
	,colframe = myexamplefr
	,coltitle = myexampleti
	,boxrule = 1pt
	,sharp corners
	,detach title
	,before upper=\tcbtitle\par\smallskip
	,fonttitle = \bfseries
	,description font = \mdseries
	,separator sign none
	,description delimiters parenthesis
}
{ex}

\newtcbtheorem[number within=chapter]{example}{Example}
{%
	colback = myexamplebg
	,breakable
	,colframe = myexamplefr
	,coltitle = myexampleti
	,boxrule = 1pt
	,sharp corners
	,detach title
	,before upper=\tcbtitle\par\smallskip
	,fonttitle = \bfseries
	,description font = \mdseries
	,separator sign none
	,description delimiters parenthesis
}
{ex}

%================================
% DEFINITION BOX
%================================

\newtcbtheorem[number within=section]{Definition}{Definition}{enhanced,
	before skip=2mm,after skip=2mm, colback=red!5,colframe=red!80!black,boxrule=0.5mm,
	attach boxed title to top left={xshift=1cm,yshift*=1mm-\tcboxedtitleheight}, varwidth boxed title*=-3cm,
	boxed title style={frame code={
					\path[fill=tcbcolback]
					([yshift=-1mm,xshift=-1mm]frame.north west)
					arc[start angle=0,end angle=180,radius=1mm]
					([yshift=-1mm,xshift=1mm]frame.north east)
					arc[start angle=180,end angle=0,radius=1mm];
					\path[left color=tcbcolback!60!black,right color=tcbcolback!60!black,
						middle color=tcbcolback!80!black]
					([xshift=-2mm]frame.north west) -- ([xshift=2mm]frame.north east)
					[rounded corners=1mm]-- ([xshift=1mm,yshift=-1mm]frame.north east)
					-- (frame.south east) -- (frame.south west)
					-- ([xshift=-1mm,yshift=-1mm]frame.north west)
					[sharp corners]-- cycle;
				},interior engine=empty,
		},
	fonttitle=\bfseries,
	title={#2},#1}{def}
\newtcbtheorem[number within=chapter]{definition}{Definition}{enhanced,
	before skip=2mm,after skip=2mm, colback=red!5,colframe=red!80!black,boxrule=0.5mm,
	attach boxed title to top left={xshift=1cm,yshift*=1mm-\tcboxedtitleheight}, varwidth boxed title*=-3cm,
	boxed title style={frame code={
					\path[fill=tcbcolback]
					([yshift=-1mm,xshift=-1mm]frame.north west)
					arc[start angle=0,end angle=180,radius=1mm]
					([yshift=-1mm,xshift=1mm]frame.north east)
					arc[start angle=180,end angle=0,radius=1mm];
					\path[left color=tcbcolback!60!black,right color=tcbcolback!60!black,
						middle color=tcbcolback!80!black]
					([xshift=-2mm]frame.north west) -- ([xshift=2mm]frame.north east)
					[rounded corners=1mm]-- ([xshift=1mm,yshift=-1mm]frame.north east)
					-- (frame.south east) -- (frame.south west)
					-- ([xshift=-1mm,yshift=-1mm]frame.north west)
					[sharp corners]-- cycle;
				},interior engine=empty,
		},
	fonttitle=\bfseries,
	title={#2},#1}{def}



%================================
% Solution BOX
%================================

\makeatletter
\newtcbtheorem{question}{Question}{enhanced,
	breakable,
	colback=white,
	colframe=myb!80!black,
	attach boxed title to top left={yshift*=-\tcboxedtitleheight},
	fonttitle=\bfseries,
	title={#2},
	boxed title size=title,
	boxed title style={%
			sharp corners,
			rounded corners=northwest,
			colback=tcbcolframe,
			boxrule=0pt,
		},
	underlay boxed title={%
			\path[fill=tcbcolframe] (title.south west)--(title.south east)
			to[out=0, in=180] ([xshift=5mm]title.east)--
			(title.center-|frame.east)
			[rounded corners=\kvtcb@arc] |-
			(frame.north) -| cycle;
		},
	#1
}{def}
\makeatother

%================================
% SOLUTION BOX
%================================

\makeatletter
\newtcolorbox{solution}{enhanced,
	breakable,
	colback=white,
	colframe=myg!80!black,
	attach boxed title to top left={yshift*=-\tcboxedtitleheight},
	title=Solution,
	boxed title size=title,
	boxed title style={%
			sharp corners,
			rounded corners=northwest,
			colback=tcbcolframe,
			boxrule=0pt,
		},
	underlay boxed title={%
			\path[fill=tcbcolframe] (title.south west)--(title.south east)
			to[out=0, in=180] ([xshift=5mm]title.east)--
			(title.center-|frame.east)
			[rounded corners=\kvtcb@arc] |-
			(frame.north) -| cycle;
		},
}
\makeatother

%================================
% Question BOX
%================================

\makeatletter
\newtcbtheorem{qstion}{Question}{enhanced,
	breakable,
	colback=white,
	colframe=mygr,
	attach boxed title to top left={yshift*=-\tcboxedtitleheight},
	fonttitle=\bfseries,
	title={#2},
	boxed title size=title,
	boxed title style={%
			sharp corners,
			rounded corners=northwest,
			colback=tcbcolframe,
			boxrule=0pt,
		},
	underlay boxed title={%
			\path[fill=tcbcolframe] (title.south west)--(title.south east)
			to[out=0, in=180] ([xshift=5mm]title.east)--
			(title.center-|frame.east)
			[rounded corners=\kvtcb@arc] |-
			(frame.north) -| cycle;
		},
	#1
}{def}
\makeatother

\newtcbtheorem[number within=chapter]{wconc}{Wrong Concept}{
	breakable,
	enhanced,
	colback=white,
	colframe=myr,
	arc=0pt,
	outer arc=0pt,
	fonttitle=\bfseries\sffamily\large,
	colbacktitle=myr,
	attach boxed title to top left={},
	boxed title style={
			enhanced,
			skin=enhancedfirst jigsaw,
			arc=3pt,
			bottom=0pt,
			interior style={fill=myr}
		},
	#1
}{def}



%================================
% NOTE BOX
%================================

\usetikzlibrary{arrows,calc,shadows.blur}
\tcbuselibrary{skins}
\newtcolorbox{note}[1][]{%
	enhanced jigsaw,
	colback=gray!20!white,%
	colframe=gray!80!black,
	size=small,
	boxrule=1pt,
	title=\textbf{Note:-},
	halign title=flush center,
	coltitle=black,
	breakable,
	drop shadow=black!50!white,
	attach boxed title to top left={xshift=1cm,yshift=-\tcboxedtitleheight/2,yshifttext=-\tcboxedtitleheight/2},
	minipage boxed title=1.5cm,
	boxed title style={%
			colback=white,
			size=fbox,
			boxrule=1pt,
			boxsep=2pt,
			underlay={%
					\coordinate (dotA) at ($(interior.west) + (-0.5pt,0)$);
					\coordinate (dotB) at ($(interior.east) + (0.5pt,0)$);
					\begin{scope}
						\clip (interior.north west) rectangle ([xshift=3ex]interior.east);
						\filldraw [white, blur shadow={shadow opacity=60, shadow yshift=-.75ex}, rounded corners=2pt] (interior.north west) rectangle (interior.south east);
					\end{scope}
					\begin{scope}[gray!80!black]
						\fill (dotA) circle (2pt);
						\fill (dotB) circle (2pt);
					\end{scope}
				},
		},
	#1,
}

%%%%%%%%%%%%%%%%%%%%%%%%%%%%%%
% SELF MADE COMMANDS
%%%%%%%%%%%%%%%%%%%%%%%%%%%%%%


\newcommand{\thm}[2]{\begin{Theorem}{#1}{}#2\end{Theorem}}
\newcommand{\cor}[2]{\begin{Corollary}{#1}{}#2\end{Corollary}}
\newcommand{\mlenma}[2]{\begin{Lenma}{#1}{}#2\end{Lenma}}
\newcommand{\mprop}[2]{\begin{Prop}{#1}{}#2\end{Prop}}
\newcommand{\clm}[3]{\begin{claim}{#1}{#2}#3\end{claim}}
\newcommand{\wc}[2]{\begin{wconc}{#1}{}\setlength{\parindent}{1cm}#2\end{wconc}}
\newcommand{\thmcon}[1]{\begin{Theoremcon}{#1}\end{Theoremcon}}
\newcommand{\ex}[2]{\begin{Example}{#1}{}#2\end{Example}}
\newcommand{\dfn}[2]{\begin{Definition}[colbacktitle=red!75!black]{#1}{}#2\end{Definition}}
\newcommand{\dfnc}[2]{\begin{definition}[colbacktitle=red!75!black]{#1}{}#2\end{definition}}
\newcommand{\qs}[2]{\begin{question}{#1}{}#2\end{question}}
\newcommand{\pf}[2]{\begin{myproof}[#1]#2\end{myproof}}
\newcommand{\nt}[1]{\begin{note}#1\end{note}}

\newcommand*\circled[1]{\tikz[baseline=(char.base)]{
		\node[shape=circle,draw,inner sep=1pt] (char) {#1};}}
\newcommand\getcurrentref[1]{%
	\ifnumequal{\value{#1}}{0}
	{??}
	{\the\value{#1}}%
}
\newcommand{\getCurrentSectionNumber}{\getcurrentref{section}}
\newenvironment{myproof}[1][\proofname]{%
	\proof[\bfseries #1: ]%
}{\endproof}

\newcommand{\mclm}[2]{\begin{myclaim}[#1]#2\end{myclaim}}
\newenvironment{myclaim}[1][\claimname]{\proof[\bfseries #1: ]}{}

\newcounter{mylabelcounter}

\makeatletter
\newcommand{\setword}[2]{%
	\phantomsection
	#1\def\@currentlabel{\unexpanded{#1}}\label{#2}%
}
\makeatother




\tikzset{
	symbol/.style={
			draw=none,
			every to/.append style={
					edge node={node [sloped, allow upside down, auto=false]{$#1$}}}
		}
}


% deliminators
\DeclarePairedDelimiter{\abs}{\lvert}{\rvert}
\DeclarePairedDelimiter{\norm}{\lVert}{\rVert}

\DeclarePairedDelimiter{\ceil}{\lceil}{\rceil}
\DeclarePairedDelimiter{\floor}{\lfloor}{\rfloor}
\DeclarePairedDelimiter{\round}{\lfloor}{\rceil}

\newsavebox\diffdbox
\newcommand{\slantedromand}{{\mathpalette\makesl{d}}}
\newcommand{\makesl}[2]{%
\begingroup
\sbox{\diffdbox}{$\mathsurround=0pt#1\mathrm{#2}$}%
\pdfsave
\pdfsetmatrix{1 0 0.2 1}%
\rlap{\usebox{\diffdbox}}%
\pdfrestore
\hskip\wd\diffdbox
\endgroup
}
\newcommand{\dd}[1][]{\ensuremath{\mathop{}\!\ifstrempty{#1}{%
\slantedromand\@ifnextchar^{\hspace{0.2ex}}{\hspace{0.1ex}}}%
{\slantedromand\hspace{0.2ex}^{#1}}}}
\ProvideDocumentCommand\dv{o m g}{%
  \ensuremath{%
    \IfValueTF{#3}{%
      \IfNoValueTF{#1}{%
        \frac{\dd #2}{\dd #3}%
      }{%
        \frac{\dd^{#1} #2}{\dd #3^{#1}}%
      }%
    }{%
      \IfNoValueTF{#1}{%
        \frac{\dd}{\dd #2}%
      }{%
        \frac{\dd^{#1}}{\dd #2^{#1}}%
      }%
    }%
  }%
}
\providecommand*{\pdv}[3][]{\frac{\partial^{#1}#2}{\partial#3^{#1}}}
%  - others
\DeclareMathOperator{\Lap}{\mathcal{L}}
\DeclareMathOperator{\Var}{Var} % varience
\DeclareMathOperator{\Cov}{Cov} % covarience
\DeclareMathOperator{\E}{E} % expected

% Since the amsthm package isn't loaded

% I prefer the slanted \leq
\let\oldleq\leq % save them in case they're every wanted
\let\oldgeq\geq
\renewcommand{\leq}{\leqslant}
\renewcommand{\geq}{\geqslant}

% % redefine matrix env to allow for alignment, use r as default
% \renewcommand*\env@matrix[1][r]{\hskip -\arraycolsep
%     \let\@ifnextchar\new@ifnextchar
%     \array{*\c@MaxMatrixCols #1}}


%\usepackage{framed}
%\usepackage{titletoc}
%\usepackage{etoolbox}
%\usepackage{lmodern}


%\patchcmd{\tableofcontents}{\contentsname}{\sffamily\contentsname}{}{}

%\renewenvironment{leftbar}
%{\def\FrameCommand{\hspace{6em}%
%		{\color{myyellow}\vrule width 2pt depth 6pt}\hspace{1em}}%
%	\MakeFramed{\parshape 1 0cm \dimexpr\textwidth-6em\relax\FrameRestore}\vskip2pt%
%}
%{\endMakeFramed}

%\titlecontents{chapter}
%[0em]{\vspace*{2\baselineskip}}
%{\parbox{4.5em}{%
%		\hfill\Huge\sffamily\bfseries\color{myred}\thecontentspage}%
%	\vspace*{-2.3\baselineskip}\leftbar\textsc{\small\chaptername~\thecontentslabel}\\\sffamily}
%{}{\endleftbar}
%\titlecontents{section}
%[8.4em]
%{\sffamily\contentslabel{3em}}{}{}
%{\hspace{0.5em}\nobreak\itshape\color{myred}\contentspage}
%\titlecontents{subsection}
%[8.4em]
%{\sffamily\contentslabel{3em}}{}{}  
%{\hspace{0.5em}\nobreak\itshape\color{myred}\contentspage}



%%%%%%%%%%%%%%%%%%%%%%%%%%%%%%%%%%%%%%%%%%%
% TABLE OF CONTENTS
%%%%%%%%%%%%%%%%%%%%%%%%%%%%%%%%%%%%%%%%%%%

\usepackage{tikz}
\definecolor{doc}{RGB}{0,60,110}
\usepackage{titletoc}
\contentsmargin{0cm}
\titlecontents{chapter}[3.7pc]
{\addvspace{30pt}%
	\begin{tikzpicture}[remember picture, overlay]%
		\draw[fill=doc!60,draw=doc!60] (-7,-.1) rectangle (-0.9,.5);%
		\pgftext[left,x=-3.5cm,y=0.2cm]{\color{white}\Large\sc\bfseries Chapter\ \thecontentslabel};%
	\end{tikzpicture}\color{doc!60}\large\sc\bfseries}%
{}
{}
{\;\titlerule\;\large\sc\bfseries Page \thecontentspage
	\begin{tikzpicture}[remember picture, overlay]
		\draw[fill=doc!60,draw=doc!60] (2pt,0) rectangle (4,0.1pt);
	\end{tikzpicture}}%
\titlecontents{section}[3.7pc]
{\addvspace{2pt}}
{\contentslabel[\thecontentslabel]{2pc}}
{}
{\hfill\small \thecontentspage}
[]
\titlecontents*{subsection}[3.7pc]
{\addvspace{-1pt}\small}
{}
{}
{\ --- \small\thecontentspage}
[ \textbullet\ ][]

\makeatletter
\renewcommand{\tableofcontents}{%
	\chapter*{%
	  \vspace*{-20\p@}%
	  \begin{tikzpicture}[remember picture, overlay]%
		  \pgftext[right,x=15cm,y=0.2cm]{\color{doc!60}\Huge\sc\bfseries \contentsname};%
		  \draw[fill=doc!60,draw=doc!60] (13,-.75) rectangle (20,1);%
		  \clip (13,-.75) rectangle (20,1);
		  \pgftext[right,x=15cm,y=0.2cm]{\color{white}\Huge\sc\bfseries \contentsname};%
	  \end{tikzpicture}}%
	\@starttoc{toc}}
\makeatother


%%From M275 "Topology" at SJSU
\newcommand{\id}{\mathrm{id}}
\newcommand{\taking}[1]{\xrightarrow{#1}}
\newcommand{\inv}{^{-1}}

%From M170 "Introduction to Graph Theory" at SJSU
\DeclareMathOperator{\diam}{diam}
\DeclareMathOperator{\ord}{ord}
\newcommand{\defeq}{\overset{\mathrm{def}}{=}}

%From the USAMO .tex files
\newcommand{\ts}{\textsuperscript}
\newcommand{\dg}{^\circ}
\newcommand{\ii}{\item}

% % From Math 55 and Math 145 at Harvard
% \newenvironment{subproof}[1][Proof]{%
% \begin{proof}[#1] \renewcommand{\qedsymbol}{$\blacksquare$}}%
% {\end{proof}}

\newcommand{\liff}{\leftrightarrow}
\newcommand{\lthen}{\rightarrow}
\newcommand{\opname}{\operatorname}
\newcommand{\surjto}{\twoheadrightarrow}
\newcommand{\injto}{\hookrightarrow}
\newcommand{\On}{\mathrm{On}} % ordinals
\DeclareMathOperator{\img}{im} % Image
\DeclareMathOperator{\Img}{Im} % Image
\DeclareMathOperator{\coker}{coker} % Cokernel
\DeclareMathOperator{\Coker}{Coker} % Cokernel
\DeclareMathOperator{\Ker}{Ker} % Kernel
\DeclareMathOperator{\rank}{rank}
\DeclareMathOperator{\Spec}{Spec} % spectrum
\DeclareMathOperator{\Tr}{Tr} % trace
\DeclareMathOperator{\pr}{pr} % projection
\DeclareMathOperator{\ext}{ext} % extension
\DeclareMathOperator{\pred}{pred} % predecessor
\DeclareMathOperator{\dom}{dom} % domain
\DeclareMathOperator{\ran}{ran} % range
\DeclareMathOperator{\Hom}{Hom} % homomorphism
\DeclareMathOperator{\Mor}{Mor} % morphisms
\DeclareMathOperator{\End}{End} % endomorphism

\newcommand{\eps}{\epsilon}
\newcommand{\veps}{\varepsilon}
\newcommand{\ol}{\overline}
\newcommand{\ul}{\underline}
\newcommand{\wt}{\widetilde}
\newcommand{\wh}{\widehat}
\newcommand{\vocab}[1]{\textbf{\color{blue} #1}}
\providecommand{\half}{\frac{1}{2}}
\newcommand{\dang}{\measuredangle} %% Directed angle
\newcommand{\ray}[1]{\overrightarrow{#1}}
\newcommand{\seg}[1]{\overline{#1}}
\newcommand{\arc}[1]{\wideparen{#1}}
\DeclareMathOperator{\cis}{cis}
\DeclareMathOperator*{\lcm}{lcm}
\DeclareMathOperator*{\argmin}{arg min}
\DeclareMathOperator*{\argmax}{arg max}
\newcommand{\cycsum}{\sum_{\mathrm{cyc}}}
\newcommand{\symsum}{\sum_{\mathrm{sym}}}
\newcommand{\cycprod}{\prod_{\mathrm{cyc}}}
\newcommand{\symprod}{\prod_{\mathrm{sym}}}
\newcommand{\Qed}{\begin{flushright}\qed\end{flushright}}
\newcommand{\parinn}{\setlength{\parindent}{1cm}}
\newcommand{\parinf}{\setlength{\parindent}{0cm}}
% \newcommand{\norm}{\|\cdot\|}
\newcommand{\inorm}{\norm_{\infty}}
\newcommand{\opensets}{\{V_{\alpha}\}_{\alpha\in I}}
\newcommand{\oset}{V_{\alpha}}
\newcommand{\opset}[1]{V_{\alpha_{#1}}}
\newcommand{\lub}{\text{lub}}
\newcommand{\del}[2]{\frac{\partial #1}{\partial #2}}
\newcommand{\Del}[3]{\frac{\partial^{#1} #2}{\partial^{#1} #3}}
\newcommand{\deld}[2]{\dfrac{\partial #1}{\partial #2}}
\newcommand{\Deld}[3]{\dfrac{\partial^{#1} #2}{\partial^{#1} #3}}
\newcommand{\lm}{\lambda}
\newcommand{\uin}{\mathbin{\rotatebox[origin=c]{90}{$\in$}}}
\newcommand{\usubset}{\mathbin{\rotatebox[origin=c]{90}{$\subset$}}}
\newcommand{\lt}{\left}
\newcommand{\rt}{\right}
\newcommand{\bs}[1]{\boldsymbol{#1}}
\newcommand{\exs}{\exists}
\newcommand{\st}{\strut}
\newcommand{\dps}[1]{\displaystyle{#1}}

\newcommand{\sol}{\setlength{\parindent}{0cm}\textbf{\textit{Solution:}}\setlength{\parindent}{1cm} }
\newcommand{\solve}[1]{\setlength{\parindent}{0cm}\textbf{\textit{Solution: }}\setlength{\parindent}{1cm}#1 \Qed}

% Things Lie
% Things Lie
\newcommand{\kb}{\mathfrak b}
\newcommand{\kg}{\mathfrak g}
\newcommand{\kh}{\mathfrak h}
\newcommand{\kn}{\mathfrak n}
\newcommand{\ku}{\mathfrak u}
\newcommand{\kz}{\mathfrak z}
\DeclareMathOperator{\Ext}{Ext} % Ext functor
\DeclareMathOperator{\Tor}{Tor} % Tor functor
\newcommand{\gl}{\opname{\mathfrak{gl}}} % frak gl group
\renewcommand{\sl}{\opname{\mathfrak{sl}}} % frak sl group chktex 6

% More script letters etc.
\newcommand{\SA}{\mathcal A}
\newcommand{\SB}{\mathcal B}
\newcommand{\SC}{\mathcal C}
\newcommand{\SF}{\mathcal F}
\newcommand{\SG}{\mathcal G}
\newcommand{\SH}{\mathcal H}
\newcommand{\OO}{\mathcal O}

\newcommand{\SCA}{\mathscr A}
\newcommand{\SCB}{\mathscr B}
\newcommand{\SCC}{\mathscr C}
\newcommand{\SCD}{\mathscr D}
\newcommand{\SCE}{\mathscr E}
\newcommand{\SCF}{\mathscr F}
\newcommand{\SCG}{\mathscr G}
\newcommand{\SCH}{\mathscr H}

% Mathfrak primes
\newcommand{\km}{\mathfrak m}
\newcommand{\kp}{\mathfrak p}
\newcommand{\kq}{\mathfrak q}

% number sets
\newcommand{\RR}[1][]{\ensuremath{\ifstrempty{#1}{\mathbb{R}}{\mathbb{R}^{#1}}}}
\newcommand{\NN}[1][]{\ensuremath{\ifstrempty{#1}{\mathbb{N}}{\mathbb{N}^{#1}}}}
\newcommand{\ZZ}[1][]{\ensuremath{\ifstrempty{#1}{\mathbb{Z}}{\mathbb{Z}^{#1}}}}
\newcommand{\QQ}[1][]{\ensuremath{\ifstrempty{#1}{\mathbb{Q}}{\mathbb{Q}^{#1}}}}
\newcommand{\CC}[1][]{\ensuremath{\ifstrempty{#1}{\mathbb{C}}{\mathbb{C}^{#1}}}}
\newcommand{\PP}[1][]{\ensuremath{\ifstrempty{#1}{\mathbb{P}}{\mathbb{P}^{#1}}}}
\newcommand{\HH}[1][]{\ensuremath{\ifstrempty{#1}{\mathbb{H}}{\mathbb{H}^{#1}}}}
\newcommand{\FF}[1][]{\ensuremath{\ifstrempty{#1}{\mathbb{F}}{\mathbb{F}^{#1}}}}
% expected value
\newcommand{\EE}{\ensuremath{\mathbb{E}}}
\newcommand{\charin}{\text{ char }}
\DeclareMathOperator{\sign}{sign}
\DeclareMathOperator{\Aut}{Aut}
\DeclareMathOperator{\Inn}{Inn}
\DeclareMathOperator{\Syl}{Syl}
\DeclareMathOperator{\Gal}{Gal}
\DeclareMathOperator{\GL}{GL} % General linear group
\DeclareMathOperator{\SL}{SL} % Special linear group

%---------------------------------------
% BlackBoard Math Fonts :-
%---------------------------------------

%Captital Letters
\newcommand{\bbA}{\mathbb{A}}	\newcommand{\bbB}{\mathbb{B}}
\newcommand{\bbC}{\mathbb{C}}	\newcommand{\bbD}{\mathbb{D}}
\newcommand{\bbE}{\mathbb{E}}	\newcommand{\bbF}{\mathbb{F}}
\newcommand{\bbG}{\mathbb{G}}	\newcommand{\bbH}{\mathbb{H}}
\newcommand{\bbI}{\mathbb{I}}	\newcommand{\bbJ}{\mathbb{J}}
\newcommand{\bbK}{\mathbb{K}}	\newcommand{\bbL}{\mathbb{L}}
\newcommand{\bbM}{\mathbb{M}}	\newcommand{\bbN}{\mathbb{N}}
\newcommand{\bbO}{\mathbb{O}}	\newcommand{\bbP}{\mathbb{P}}
\newcommand{\bbQ}{\mathbb{Q}}	\newcommand{\bbR}{\mathbb{R}}
\newcommand{\bbS}{\mathbb{S}}	\newcommand{\bbT}{\mathbb{T}}
\newcommand{\bbU}{\mathbb{U}}	\newcommand{\bbV}{\mathbb{V}}
\newcommand{\bbW}{\mathbb{W}}	\newcommand{\bbX}{\mathbb{X}}
\newcommand{\bbY}{\mathbb{Y}}	\newcommand{\bbZ}{\mathbb{Z}}

%---------------------------------------
% MathCal Fonts :-
%---------------------------------------

%Captital Letters
\newcommand{\mcA}{\mathcal{A}}	\newcommand{\mcB}{\mathcal{B}}
\newcommand{\mcC}{\mathcal{C}}	\newcommand{\mcD}{\mathcal{D}}
\newcommand{\mcE}{\mathcal{E}}	\newcommand{\mcF}{\mathcal{F}}
\newcommand{\mcG}{\mathcal{G}}	\newcommand{\mcH}{\mathcal{H}}
\newcommand{\mcI}{\mathcal{I}}	\newcommand{\mcJ}{\mathcal{J}}
\newcommand{\mcK}{\mathcal{K}}	\newcommand{\mcL}{\mathcal{L}}
\newcommand{\mcM}{\mathcal{M}}	\newcommand{\mcN}{\mathcal{N}}
\newcommand{\mcO}{\mathcal{O}}	\newcommand{\mcP}{\mathcal{P}}
\newcommand{\mcQ}{\mathcal{Q}}	\newcommand{\mcR}{\mathcal{R}}
\newcommand{\mcS}{\mathcal{S}}	\newcommand{\mcT}{\mathcal{T}}
\newcommand{\mcU}{\mathcal{U}}	\newcommand{\mcV}{\mathcal{V}}
\newcommand{\mcW}{\mathcal{W}}	\newcommand{\mcX}{\mathcal{X}}
\newcommand{\mcY}{\mathcal{Y}}	\newcommand{\mcZ}{\mathcal{Z}}


%---------------------------------------
% Bold Math Fonts :-
%---------------------------------------

%Captital Letters
\newcommand{\bmA}{\boldsymbol{A}}	\newcommand{\bmB}{\boldsymbol{B}}
\newcommand{\bmC}{\boldsymbol{C}}	\newcommand{\bmD}{\boldsymbol{D}}
\newcommand{\bmE}{\boldsymbol{E}}	\newcommand{\bmF}{\boldsymbol{F}}
\newcommand{\bmG}{\boldsymbol{G}}	\newcommand{\bmH}{\boldsymbol{H}}
\newcommand{\bmI}{\boldsymbol{I}}	\newcommand{\bmJ}{\boldsymbol{J}}
\newcommand{\bmK}{\boldsymbol{K}}	\newcommand{\bmL}{\boldsymbol{L}}
\newcommand{\bmM}{\boldsymbol{M}}	\newcommand{\bmN}{\boldsymbol{N}}
\newcommand{\bmO}{\boldsymbol{O}}	\newcommand{\bmP}{\boldsymbol{P}}
\newcommand{\bmQ}{\boldsymbol{Q}}	\newcommand{\bmR}{\boldsymbol{R}}
\newcommand{\bmS}{\boldsymbol{S}}	\newcommand{\bmT}{\boldsymbol{T}}
\newcommand{\bmU}{\boldsymbol{U}}	\newcommand{\bmV}{\boldsymbol{V}}
\newcommand{\bmW}{\boldsymbol{W}}	\newcommand{\bmX}{\boldsymbol{X}}
\newcommand{\bmY}{\boldsymbol{Y}}	\newcommand{\bmZ}{\boldsymbol{Z}}
%Small Letters
\newcommand{\bma}{\boldsymbol{a}}	\newcommand{\bmb}{\boldsymbol{b}}
\newcommand{\bmc}{\boldsymbol{c}}	\newcommand{\bmd}{\boldsymbol{d}}
\newcommand{\bme}{\boldsymbol{e}}	\newcommand{\bmf}{\boldsymbol{f}}
\newcommand{\bmg}{\boldsymbol{g}}	\newcommand{\bmh}{\boldsymbol{h}}
\newcommand{\bmi}{\boldsymbol{i}}	\newcommand{\bmj}{\boldsymbol{j}}
\newcommand{\bmk}{\boldsymbol{k}}	\newcommand{\bml}{\boldsymbol{l}}
\newcommand{\bmm}{\boldsymbol{m}}	\newcommand{\bmn}{\boldsymbol{n}}
\newcommand{\bmo}{\boldsymbol{o}}	\newcommand{\bmp}{\boldsymbol{p}}
\newcommand{\bmq}{\boldsymbol{q}}	\newcommand{\bmr}{\boldsymbol{r}}
\newcommand{\bms}{\boldsymbol{s}}	\newcommand{\bmt}{\boldsymbol{t}}
\newcommand{\bmu}{\boldsymbol{u}}	\newcommand{\bmv}{\boldsymbol{v}}
\newcommand{\bmw}{\boldsymbol{w}}	\newcommand{\bmx}{\boldsymbol{x}}
\newcommand{\bmy}{\boldsymbol{y}}	\newcommand{\bmz}{\boldsymbol{z}}

%---------------------------------------
% Scr Math Fonts :-
%---------------------------------------

\newcommand{\sA}{{\mathscr{A}}}   \newcommand{\sB}{{\mathscr{B}}}
\newcommand{\sC}{{\mathscr{C}}}   \newcommand{\sD}{{\mathscr{D}}}
\newcommand{\sE}{{\mathscr{E}}}   \newcommand{\sF}{{\mathscr{F}}}
\newcommand{\sG}{{\mathscr{G}}}   \newcommand{\sH}{{\mathscr{H}}}
\newcommand{\sI}{{\mathscr{I}}}   \newcommand{\sJ}{{\mathscr{J}}}
\newcommand{\sK}{{\mathscr{K}}}   \newcommand{\sL}{{\mathscr{L}}}
\newcommand{\sM}{{\mathscr{M}}}   \newcommand{\sN}{{\mathscr{N}}}
\newcommand{\sO}{{\mathscr{O}}}   \newcommand{\sP}{{\mathscr{P}}}
\newcommand{\sQ}{{\mathscr{Q}}}   \newcommand{\sR}{{\mathscr{R}}}
\newcommand{\sS}{{\mathscr{S}}}   \newcommand{\sT}{{\mathscr{T}}}
\newcommand{\sU}{{\mathscr{U}}}   \newcommand{\sV}{{\mathscr{V}}}
\newcommand{\sW}{{\mathscr{W}}}   \newcommand{\sX}{{\mathscr{X}}}
\newcommand{\sY}{{\mathscr{Y}}}   \newcommand{\sZ}{{\mathscr{Z}}}


%---------------------------------------
% Math Fraktur Font
%---------------------------------------

%Captital Letters
\newcommand{\mfA}{\mathfrak{A}}	\newcommand{\mfB}{\mathfrak{B}}
\newcommand{\mfC}{\mathfrak{C}}	\newcommand{\mfD}{\mathfrak{D}}
\newcommand{\mfE}{\mathfrak{E}}	\newcommand{\mfF}{\mathfrak{F}}
\newcommand{\mfG}{\mathfrak{G}}	\newcommand{\mfH}{\mathfrak{H}}
\newcommand{\mfI}{\mathfrak{I}}	\newcommand{\mfJ}{\mathfrak{J}}
\newcommand{\mfK}{\mathfrak{K}}	\newcommand{\mfL}{\mathfrak{L}}
\newcommand{\mfM}{\mathfrak{M}}	\newcommand{\mfN}{\mathfrak{N}}
\newcommand{\mfO}{\mathfrak{O}}	\newcommand{\mfP}{\mathfrak{P}}
\newcommand{\mfQ}{\mathfrak{Q}}	\newcommand{\mfR}{\mathfrak{R}}
\newcommand{\mfS}{\mathfrak{S}}	\newcommand{\mfT}{\mathfrak{T}}
\newcommand{\mfU}{\mathfrak{U}}	\newcommand{\mfV}{\mathfrak{V}}
\newcommand{\mfW}{\mathfrak{W}}	\newcommand{\mfX}{\mathfrak{X}}
\newcommand{\mfY}{\mathfrak{Y}}	\newcommand{\mfZ}{\mathfrak{Z}}
%Small Letters
\newcommand{\mfa}{\mathfrak{a}}	\newcommand{\mfb}{\mathfrak{b}}
\newcommand{\mfc}{\mathfrak{c}}	\newcommand{\mfd}{\mathfrak{d}}
\newcommand{\mfe}{\mathfrak{e}}	\newcommand{\mff}{\mathfrak{f}}
\newcommand{\mfg}{\mathfrak{g}}	\newcommand{\mfh}{\mathfrak{h}}
\newcommand{\mfi}{\mathfrak{i}}	\newcommand{\mfj}{\mathfrak{j}}
\newcommand{\mfk}{\mathfrak{k}}	\newcommand{\mfl}{\mathfrak{l}}
\newcommand{\mfm}{\mathfrak{m}}	\newcommand{\mfn}{\mathfrak{n}}
\newcommand{\mfo}{\mathfrak{o}}	\newcommand{\mfp}{\mathfrak{p}}
\newcommand{\mfq}{\mathfrak{q}}	\newcommand{\mfr}{\mathfrak{r}}
\newcommand{\mfs}{\mathfrak{s}}	\newcommand{\mft}{\mathfrak{t}}
\newcommand{\mfu}{\mathfrak{u}}	\newcommand{\mfv}{\mathfrak{v}}
\newcommand{\mfw}{\mathfrak{w}}	\newcommand{\mfx}{\mathfrak{x}}
\newcommand{\mfy}{\mathfrak{y}}	\newcommand{\mfz}{\mathfrak{z}}
 

\title{Special Functions in Calculus}
\author{{Gerald Cainicela}}
\date{\today}
\usepackage{physics}
\usepackage{xcolor} % Requerido para reglas horizontales más agradables en tablas
\definecolor{ocre}{HTML}{000AFF} % Definir el color naranja usado para resaltar en todo el libro.

\newcommand{\resolucion}{ \begin{center}
		{\bf\color{ocre} Resolución}
\end{center}}

\usepackage{fancyhdr}
\usepackage{hyperref}

\pagestyle{fancy}
\fancyhf{} % Limpiar encabezado y pie

% Encabezados
\fancyhead[L]{} % Izquierda
\fancyhead[C]{\textbf{Special Functions in Calculus}} % Centro
\fancyhead[R]{\thepage} % Derecha

% Pie de página
\fancyfoot[L]{\href{https://github.com/asdcainicela/special-functions-calculus}{GitHub}} % Izquierda
\fancyfoot[C]{}
\fancyfoot[R]{\href{https://www.linkedin.com/in/gerald-cainicela/}{LinkedIn}} % Derecha

\begin{document}

\maketitle
\newpage% or \cleardoublepage
% \pdfbookmark[<level>]{<title>}{<dest>}
\pdfbookmark[section]{\contentsname}{toc}
 %\tableofcontents
\pagebreak

\chapter{Familias del Factorial}

\section{Multifactorial y Familias del Factorial}

\subsection{Introducción}

De manera fundamental el factorial de n representa el número de formas distintas de ordenar n objetos distintos (elementos sin repetición). Este hecho ha sido conocido desde hace varios siglos, en el siglo XII por los estudiosos hindúes.

\subsection{Notaciones Tradicionales}

\begin{multicols}{2}


	\begin{tabular}{||c||c||}
		\hline
		Función         & Notaciones Tradicionales \\
		\hline
		Factorial       & $n!$                     \\
		\hline
		Doble Factorial & $n!!$                    \\
		\hline
		Multifactorial  & $n!^{k}$                 \\
		\hline
		Primorial       & $ n\# $                  \\
		\hline
	\end{tabular}


	\begin{tabular}{||c||c||}
		\hline
		Función                   & Notaciones  Tradicionales \\
		\hline
		Superfactorial            & $Sf(n), G$                \\
		\hline
		Pickover’s Superfactorial & $n\textdollar$            \\
		\hline
		Hiperfactoria             & $H(n)$                    \\
		\hline
	\end{tabular}

\end{multicols}


\section{Definiciones}\index{Definición}


\subsection{Factorial}

\dfn{Definición}{
	\[n!=\left\{ \begin{array}{rcl}
			n(n-1)(n-2)\ldots 3\cdot2\cdot1 &  & n=1, 2, 3,\ldots \\
			                                &  &                  \\
			1                               &  & n=0              \\
		\end{array}
		\right. \]

}



\subsection{Doble Factorial}

\dfn{Definición}{
	\[n!!=\left\{ \begin{array}{rcl}
			n(n-2)(n-4)\ldots 5\cdot3\cdot1 &  & ; n>0 \ \ \ Impar \\
			                                &  &                   \\
			n(n-2)(n-4)\ldots 6\cdot4\cdot2 &  & ; n>0 \ \ \  Par  \\
			                                &  &                   \\
			1                               &  & ; n=-1,0          \\
		\end{array}
		\right. \]
}



\subsection{Multifactorial}

\dfn{Definición}{

\[n!^{k}=\left\{ \begin{array}{rcl}
		n((n-k)!^{k}) &  & ;   \ n> k      \\
		              &  &                 \\
		n             &  & ;   \ 0<n\leq k \\
		              &  &                 \\
		1             &  & ;  \ -k<n\leq 0
	\end{array}
	\right. \]


}


\subsection{Primorial}

\dfn{Definición}{

	\[ n\# =\left\{ \begin{array}{rcl}
			(n-1)\#\times n &  & ;   \ n \ \  \mbox{es primo}   \\
			                &  &                                \\
			(n-1)\#         &  & ;   \ n \ \ \mbox{no es primo} \\
			                &  &                                \\
			1               &  & ;  \ n=0,1
		\end{array}
		\right. \]
}

\subsection{Superfactorial}

\dfn{Definición}{

	\[  \mbox{sf}(n)=\left\{ \begin{array}{rcl}
			 &  & \displaystyle \prod _{k=1}^{n}k!                           \\
			 &  &                                                            \\
			 &  & \displaystyle \prod _{k=1}^{n}k^{n-k+1}                    \\
			 &  &                                                            \\
			 &  & 1^{n}\cdot 2^{n-1}\cdot 3^{n-2}\cdots (n-1)^{2}\cdot n^{1}
		\end{array}
		\right. \]

}


\subsection{Pickover’s Superfactorial}


$$\displaystyle n\$ \equiv \underbrace{n!^{{n!}^{{\cdot }^{{\cdot }^{{\cdot }^{n!}}}}}}_{n!}$$


\dfn{Definición}{

	\[  n\$ =\left\{ \begin{array}{rcl}
			 &  & \underbrace{n!^{{n!}^{{\cdot }^{{\cdot }^{{\cdot }^{n!}}}}}}_{n!} \\
			 &  &                                                                   \\
			 &  & {}^{n!}(n!)                                                       \\
			 &  &                                                                   \\
			 &  & (n!)\uparrow \uparrow (n!)=(n!)\uparrow \uparrow \uparrow 2       \\
		\end{array}
		\right. \]

}

\subsection{Hiperfactorial}



\dfn{Definición}{

	\[  H(n) =\left\{ \begin{array}{rcl}
			 &  & \displaystyle \prod _{k=1}^{n}k^{k}                      \\
			 &  &                                                          \\
			 &  & 1^{1}\cdot 2^{2}\cdot 3^{3}\cdots (n-1)^{n-1}\cdot n^{n} \\
		\end{array}
		\right. \]

}




\section{Representaciones y Propiedades}\index{Representaciones y Propiedades}



\subsection{Representaciones con Integrales}


\begin{align*}
	n!= & \int_{0}^{\infty} t^n e^{-t} \; \mbox{d}t \hspace{0.5cm} ;\hspace{.5cm}  n\in\mathbb{N}
	\\
	n!= & \int_{0}^{\infty} \left( e^{-t} - \sum_{k=0}^{m}\dfrac{(-t)^k}{k!} \right)t^{n} \; \mbox{d}t \hspace{0.5cm} ;\hspace{.5cm}  m\in\mathbb{N}^+ \wedge -m-1<\Re(n)<-m
	\\
	n!= & \int_{0}^{1} \ln^n\left(\dfrac{1}{t} \right)  \; \mbox{d}t \hspace{0.5cm} ;\hspace{.5cm} \Re(n)>-1
\end{align*}

\subsection{Representaciones con Límites}

\begin{align*}
	n!= & \lim\limits_{x \to 1} \dfrac{(1-x)^{n-1}}{ \prod_{k=2}^{n} (1-x^{1/k})}\hspace{0.5cm} ; \hspace{.5cm} n\in\mathbb{N}
	\\
	n!= & \lim\limits_{p \to \infty} \int_{0}^{\infty} \left(1-\dfrac{x}{p} \right)^{p } x^p \; \mbox{d}x   \hspace{0.5cm} ; \hspace{.5cm} \Re(n)>-1
\end{align*}





\subsection{Representaciones de Producto Factorial}
\begin{align*}
	 & n!=\prod_{k=1}^{n} k \hspace{.5cm} \ \ ; \hspace{.5cm} n\in\mathbb{N}                                                                                                                                                          \\
	 & \left( n+\dfrac{a}{b}\right) !=\dfrac{1}{b^n} \left(\dfrac{a}{b} \right)! \prod_{k=1}^{n} (a+kb) \hspace{0.5cm}\ \ ; \hspace{.5cm} n\in\mathbb{N} \wedge a \in\mathbb{N}^+ \wedge b \in\mathbb{N}^+ \wedge a<b
	\\
	 &
	\\
	 & \left( \dfrac{a}{b}-n\right) != \left(\dfrac{a}{b} \right)! \dfrac{(-1)^n  b^n}{ \prod_{k=1}^{n} (-a+kb-b)  }    \hspace{0.5cm}\ \ ; \hspace{.5cm} n\in\mathbb{N} \wedge a \in\mathbb{N}^+ \wedge b \in\mathbb{N}^+ \wedge a<b
	\\
	 &
	\\
	 & n!= \dfrac{1}{n+1}\prod_{k=1}^{\infty} \dfrac{\left(1+\dfrac{1}{k} \right)^{n+1} }{\dfrac{n+1}{k}+1} \hspace{0.5cm}\ \ ; \hspace{.5cm} n \notin \mathbb{N}
\end{align*}

\section{Tabla de Valores Notables}\index{Tabla de Valores Notables}

\begin{tabular}{|c|c|c|c|c|c|c|c|c|c|c|c|c|}
	\hline
	n       & -1  & 0 & 1 & 2 & 3 & 4  & 5   & 6   & 7    & 8     & 9      & 10      \\
	\hline
	$$ n!$$ & N.A & 1 & 1 & 2 & 6 & 24 & 120 & 720 & 5040 & 40320 & 362880 & 3628800 \\
	\hline
\end{tabular}

\subsection{Relación del Factorial con el Doble Factorial}
\begin{align*}
	n!= & 2^{-(\sin^2(-\pi n))/2-n} \pi^{(\sin^2(n\pi))/2} (2n)!! \\
	n!= & (n-1)! n!!
\end{align*}


\subsection{Multiple argumento}

\begin{align*}
	(2n)!= & \dfrac{2^{2n} n }{\sqrt{\pi}} (n-1)!\left( n-\dfrac{1}{2}\right)!
	\\
	(3n)!= & \dfrac{3^{3n+1/2} n }{2\pi} (n-1)!\left( n-\dfrac{2}{3}\right)! \left( n-\dfrac{1}{3}\right)!
	\\
	(mn)!= & nm^{mn+1/2} (2 \pi)^{(1-m)/2}  \prod_{k=0}^{m-1} \left(  \dfrac{k}{m}+n-1\right) ! \hspace{0.5cm}\ \ ; \hspace{.5cm} m \in \mathbb{N}
\end{align*}

































\section{factoriales}
\subsection{teoria}

Multifactorial

su simbolo es $n!^{k} $


\begin{align*}
	n!^{k} = & n((n-k)!^{k}), si  \ n> k \\
	n!^{k} = & n, si  \ 0<n\leq k        \\
	n!^{k} = & 1, si  \ -k<n\leq 0
\end{align*}

Propiedad

$$
	n!=\prod_{i=0}^{k-1} (n-i)!^{(k)}, para k \in \mathbf{z}^+ , n\geq k-1
$$


Asimismo tambien hay otra representación

su simbolo es $n!_{k} $


\begin{align*}
	n!_{k} = & n((n-k)!_{k}), si  \ n> 0 \\
	n!^{k} = & 1, si  \ -k<n\leq 0       \\
	n!^{k} = & 0, si  \ otroscasos
\end{align*}

$$
	n!_{k} =n(n-k)...(k+1)=\left(k \right)^{(n-1)/(k)} \dfrac{\Gamma (n/(k)+1)}{\Gamma(1/k+1) }
$$

$${\displaystyle n!_{(\alpha )}={\begin{cases}n\cdot (n-\alpha )!_{(\alpha )}&{\text{ if }}n>0\,;\\1&{\text{ if }}-\alpha <n\leq 0\,;\\0&{\text{ otherwise. }}\end{cases}}}$$

$${\displaystyle {\begin{aligned}n!_{(\alpha )}&=n(n-\alpha )\cdots (\alpha +1)\\&=\alpha ^{\frac {n-1}{\alpha }}\left({\frac {n}{\alpha }}\right)\left({\frac {n-\alpha }{\alpha }}\right)\cdots \left({\frac {\alpha +1}{\alpha }}\right)\\&=\alpha ^{\frac {n-1}{\alpha }}{\frac {\Gamma \left({\frac {n}{\alpha }}+1\right)}{\Gamma \left({\frac {1}{\alpha }}+1\right)}}\,.\end{aligned}}}$$

Propiedades

$$\displaystyle {\begin{aligned}(\alpha n-1)!_{(\alpha )}&=\sum _{k=0}^{n-1}{\binom {n-1}{k+1}}(-1)^{k}\times \left({\frac {1}{\alpha }}\right)_{-(k+1)}\left({\frac {1}{\alpha }}-n\right)_{k+1}\times {\bigl (}\alpha (k+1)-1{\bigr )}!_{(\alpha )}{\bigl (}\alpha (n-k-1)-1{\bigr )}!_{(\alpha )}\\&=\sum _{k=0}^{n-1}{\binom {n-1}{k+1}}(-1)^{k}\times {\binom {{\frac {1}{\alpha }}+k-n}{k+1}}{\binom {{\frac {1}{\alpha }}-1}{k+1}}\times {\bigl (}\alpha (k+1)-1{\bigr )}!_{(\alpha )}{\bigl (}\alpha (n-k-1)-1{\bigr )}!_{(\alpha )}\,,\end{aligned}}$$

$$\displaystyle {\begin{aligned}(\alpha n-1)!_{(\alpha )}&=\sum _{k=0}^{n-1}\sum _{i=0}^{k+1}{\binom {n-1}{k+1}}{\binom {k+1}{i}}(-1)^{k}\alpha ^{k+1-i}(\alpha i-1)!_{(\alpha )}{\bigl (}\alpha (n-1-k)-1{\bigr )}!_{(\alpha )}\times (n-1-k)_{k+1-i}\\&=\sum _{k=0}^{n-1}\sum _{i=0}^{k+1}{\binom {n-1}{k+1}}{\binom {k+1}{i}}{\binom {n-1-i}{k+1-i}}(-1)^{k}\alpha ^{k+1-i}(\alpha i-1)!_{(\alpha )}{\bigl (}\alpha (n-1-k)-1{\bigr )}!_{(\alpha )}\times (k+1-i)!.\end{aligned}}$$


como un caso particular

$${\displaystyle (2n-1)!!=\sum _{k=0}^{n-1}{\binom {n}{k+1}}(2k-1)!!(2n-2k-3)!!.}$$



doblde factorial

$$ {\displaystyle n!!={\frac {(2k)!}{2^{k}k!}}={\frac {(2k-1)!}{2^{k-1}(k-1)!}}\,.}$$

$${\displaystyle (2n-1)!!=2^{n}\cdot {\frac {\Gamma \left({\frac {1}{2}}+n\right)}{\sqrt {\pi }}}=(-2)^{n}\cdot {\frac {\sqrt {\pi }}{\Gamma \left({\frac {1}{2}}-n\right)}}\,.}$$

$${\displaystyle {\begin{aligned}(2n-1)!!&=\sum _{k=1}^{n-1}{\binom {n}{k+1}}(2k-1)!!(2n-2k-3)!!\,,\\(2n-1)!!&=\sum _{k=0}^{n}{\binom {2n-k-1}{k-1}}{\frac {(2k-1)(2n-k+1)}{k+1}}(2n-2k-3)!!\,,\\(2n-1)!!&=\sum _{k=1}^{n}{\frac {(n-1)!}{(k-1)!}}k(2k-3)!!\,.\end{aligned}}}$$

$${\displaystyle {\frac {(2n)!!}{(2n-1)!!}}\approx {\sqrt {\pi n}}.}$$


Primorial

denotado n,es similar al factorial, pero con el producto tomado sólo sobre los números primos menores o iguales a n. Es decir

$${\displaystyle n\#=\prod _{p\leq n}p,}$$

Superfactorial
Neil Sloane y Simon Plouffe definieron un superfactorial en The Encyclopedia of Integer Sequences (Academic Press, 1995) como el producto de los primeros n factoriales. Así que el superfactorial de 4 es





\subsection{Problemas Resueltos}

\ex{ejemplo}{Encuentre el valor del límite para todo $n =0,1,2,\ldots$
	$$
		\lim\limits_{x \to \infty} \dfrac{(x+n+1)!}{(x+n)!+(x+n+1)!}
	$$

	\resolucion


	Usamos $(a+1)=(a+1)a!$

	$$
		\lim\limits_{x \to \infty} \dfrac{(x+n+1)(x+n)!}{(x+n)!+(x+n+1)(x+n)!}
	$$

	Factorizamos terminos semejantes

	$$
		\lim\limits_{x \to \infty} \dfrac{(x+n+1)}{\left[1+(x+n+1) \right]}\cdot \cancelto{1}{\dfrac{(x+n)!}{(x+n)!}}
		=
		\lim\limits_{x \to \infty} \dfrac{x+n+1}{x+n+2}
	$$

	$$
		\therefore \lim\limits_{x \to \infty} \dfrac{(x+ n+1)!}{(x+n)!+(xn+1)!}=1
	$$
}




\ex{ejemplo}{Encuentre el valor del límite si  $k=1,2,3,\ldots$


	$$
		\lim\limits_{n \to \infty} \dfrac{\ln(n^n)}{\ln((kn)!)}
	$$

	\resolucion



	Usamos el Criterio de Stolz-Cesàro

	$$
		\lim\limits_{n \to \infty} \dfrac{\ln(n^n)}{\ln((2n)!)}=L \Rightarrow  \lim\limits_{n \to \infty} \dfrac{\ln((n+1)^{(n+1)})-\ln(n^n)}{\ln((kn+k)!)-\ln((kn)!)}=L
	$$

	Agrupamos terminos semejantes

	$$
		L=\lim\limits_{n \to \infty} \dfrac{n \left( \ln \left( n+1\right) -\ln(n)\right)   +\ln(n+1)}{\ln\left( \dfrac{(kn+k)!}{(kn)!}\right) }
	$$

	Sabemos que $\ln(a/b)=\ln(a)-\ln(b) ; a,b>0 ; a\neq 1$

	$$
		L=\lim\limits_{n \to \infty} \dfrac{n\ln \left( \dfrac{n+1}{n}\right)  +\ln(n+1)}{\ln\left( \dfrac{(kn+k)!}{(kn)!}\right) }
	$$

	Es fácil comprobar que $\lim_{n \to \infty}\ln \left( \frac{n+1}{n}\right)^n =1$

	$$
		L=\lim\limits_{n \to \infty} \dfrac{\ln \left( \dfrac{n+1}{n}\right)^n  +\ln(n+1)}{\ln\left( (kn+k)(kn+k-1)\ldots(kn+1)\right) }
	$$

	Separamos el límite  y se observa queel primero es igual a 0

	$$
		L= \lim\limits_{n \to \infty} \dfrac{1}{\ln\left( (kn+k)(kn+k-1)\ldots(kn+1)\right) }+ \lim\limits_{n \to \infty} \dfrac{\ln(n+1)}{\ln\left( (kn+k)(kn+k-1)\ldots(kn+1)\right) }
	$$

	$$
		L=  \lim\limits_{n \to \infty} \dfrac{\ln(n+1)}{\ln\left( (kn)^{k}+ \ldots +k!\right) }
	$$

	Sean $C_2, C_3, \ldots , C_{k-2}, C_{k-1}, C_k$ constantes
	$$
		L=\lim\limits_{n \to \infty} \dfrac{\ln(n+1)}{\ln\left( (kn)^{k} \left( 1+\dfrac{C_{2} n^{k-1}}{(kn)^{k}}  \ldots+\dfrac{C_{k-1} n}{(kn)^{k}} +\dfrac{k!}{(kn)^{k}} \right) \right) }
	$$

	$$
		L=\lim\limits_{n \to \infty} \dfrac{\ln(n+1)}{\ln\left( (kn)^{k} \left( 1+\cancelto{0}{\dfrac{C_{2} n^{k-1}}{(kn)^{k}}}  \ldots+ \cancelto{0}{\dfrac{C_{k-1} n}{(kn)^{k}}} +\cancelto{0}{\dfrac{k!}{(kn)^{k}}} \hspace{0.5cm}\right) \right) }
	$$

	$$
		L=\lim\limits_{n \to \infty} \dfrac{\ln(n+1)}{\ln\left( (kn)^{k} \right) }=\lim\limits_{n \to \infty} \dfrac{\ln(n+1)}{k\ln\left(kn \right) }=\dfrac{1}{k}
	$$


	$$
		\therefore \lim\limits_{n \to \infty} \dfrac{\ln(n^n)}{\ln((kn)!)}=\dfrac{1}{k}
	$$
}



\chapter{Función Gamma (Factorial)}
\section{Definición}
\dfn{Definición}{
	La función gamma de Euler se define   por la integral impropia. 	Si $\Re(z)>-1$ entonces, la Función Gamma converge absolutamente:

	\[
		\Gamma(z) := \int_0^\infty t^{z-1} e^{-t} \, \dd{t}
	\]

	$$
		\Gamma(z+1)=\int_0^{\infty} t^{z } e^{-t}  \dd{t} , \quad z \neq -1,-2,-3,\ldots
	$$

	Es una extensión del factorial, pues para \( n \in \mathbb{N} \):

	\[
		\Gamma(n) = (n-1)!
	\]

}

También puede ser equivalente a:
$$
	\Gamma(z) := x^z \int_0^\infty t^{z-1} e^{-kt} \, \dd{t}, \quad (\Re z >0, \Re k >0)
$$

\ex{}{
	$$
		\int_{0}^{\infty} t^{i+1}  e^{-t}dt=\int_{0}^{\infty} t^{(i+2)-1} e^{-t}dt=\Gamma(i+2)
	$$
	$$
		\int_{0}^{\infty} t^{2} e^{-t}dt=\int_{0}^{\infty} t^{3-1} e^{-t}dt=\Gamma(3)=2!=2
	$$
	$$
		\int_{0}^{\infty} t^{5/9} e^{-t}dt=\int_{0}^{\infty} t^{14/9-1} e^{-t}dt=\Gamma\left(\frac{14}{9} \right)
	$$
}
\mprop{}{	Valor en 1: \( \Gamma(1) = 1 \).}

\begin{proof}
	Usamos la definición de la función gamma de Euler, sustituyendo $z = 1$, se tiene:
	\[
		\Gamma(1) = \int_0^{\infty} t^{1-1} e^{-t} \, \dd{t} = \int_0^{\infty} e^{-t} \, \dd{t} = \left[ -e^{-t} \right]_0^{\infty} = 0 - (-1) = 1.
	\]
	Por lo tanto,
	\[
		\Gamma(1) = 1. \qedhere
	\]
\end{proof}

\mprop{Relación funcional de la función Gamma}{
	$$\Gamma(z+1) = z \, \Gamma(z) $$
}

\begin{proof}
	Usamos la definición:
	\[
		\Gamma(z+1) = \int_0^\infty t^z e^{-t} \, \dd{t}.
	\]
	Aplicamos integración por partes con:
	\[
		u = t^z \Rightarrow du = z t^{z-1} \dd{t}, \quad dv = e^{-t} \dd{t} \Rightarrow v = -e^{-t}.
	\]
	Entonces:
	\[
		\Gamma(z+1) = \left[ -t^z e^{-t} \right]_0^\infty + \int_0^\infty z t^{z-1} e^{-t} \, \dd{t}.
	\]
	El término evaluado en los extremos es cero, ya que:
	\[
		\lim_{t \to \infty} t^z e^{-t} = 0 \quad \text{y} \quad \lim_{t \to 0^+} t^z e^{-t} = 0 \quad \text{para } \operatorname{Re}(z) > 0.
	\]
	Por tanto,
	\[
		\Gamma(z+1) = z \int_0^\infty t^{z-1} e^{-t} \, \dd{t} = z \, \Gamma(z), \qquad   \Re(z)>-1  \text{ y } \Re(z)\neq -1,-2,-3, \cdots \qedhere
	\]
\end{proof}

\nt{De la recurrencia,
	\begin{align*}
		\Gamma(n+z) & = (n-1+z)(n-2+z) \ldots (1+z) \Gamma(1+z) \\
		            & =(n-1+z)(n-2+z) \ldots (1+z) z!           \\
		            & = (n-1+z)!
	\end{align*}}

\ex{}{	Simplifique la siguiente expresión

	$$
		\frac{\Gamma\left(\frac{5}{2} \right)\Gamma\left(\frac{9 }{7} \right)  }{\Gamma\left(\frac{1}{2} \right)\Gamma\left(\frac{2 }{7} \right) }
	$$

	\resolucion

	Sabemos por la Formula de Reducción $\Gamma(z+1)=z \Gamma(z)	$

	$$
		\Gamma\left( \frac{5}{2}\right) =\Gamma\left(1+ \frac{3}{2}\right) =\frac{3}{2}\Gamma\left(\frac{3}{2}\right) = \frac{3}{2}\Gamma\left(1+\frac{1}{2}\right)=\frac{3}{2} \frac{1}{2} \Gamma\left(\frac{1}{2}\right)
	$$

	$$
		\Gamma\left( \frac{9}{7}\right) =\Gamma\left(1+ \frac{2}{7}\right) =\frac{2}{7}\Gamma\left(\frac{2}{7}\right)
	$$

	Reemplazando en la expresión
	$$
		\frac{\Gamma\left(\frac{5}{2} \right)\Gamma\left(\frac{9 }{7} \right)  }{\Gamma\left(\frac{1}{2} \right)\Gamma\left(\frac{2 }{7} \right) }= \frac{\frac{3}{2} \cdotp \frac{1}{2} \Gamma\left(\frac{1}{2}\right) \cdotp      \frac{2}{7}\Gamma\left(\frac{2}{7}\right) }{\Gamma\left(\frac{1}{2} \right)\Gamma\left(\frac{2 }{7} \right) } =\frac{3}{2} \cdotp \frac{1}{2}\cdotp  \frac{2}{7} =\frac{3}{14}
	$$

	$$
		\therefore \frac{\Gamma\left(\frac{5}{2} \right)\Gamma\left(\frac{9 }{7} \right)  }{\Gamma\left(\frac{1}{2} \right)\Gamma\left(\frac{2 }{7} \right) }=\frac{3}{14}
	$$
}
\ex{Encuentre el valor de $\Gamma(\frac{5}{2})$ si $\Gamma(\frac{1}{2}) = \sqrt{\pi}$}{\label{ex:ejmplo_01_01}

	\begin{align*}
		\Gamma\qty(\frac{5}{2}) & = \Gamma\qty(1+\frac{3}{2})  =  \frac{3}{2} \cdot \Gamma\qty(\frac{3}{2}) =  \frac{3}{2} \cdot\pqty{ \Gamma\qty(1+\frac{1}{2}) } = \\
		                        & =  \frac{3}{2} \cdot\pqty{\frac{1}{2}\cdot \Gamma\qty(\frac{1}{2}) } = \frac{3}{4} \cdot \Gamma\qty(\frac{1}{2})                   \\
		                        & = \frac{3}{4} \sqrt{\pi}
	\end{align*}
}

\ex{}{	Calcular la siguiente integral, si $\Gamma\left( \frac{1}{2} \right)=\sqrt{\pi} $

	$$
		\int_{0}^{\infty} t^{7/2} e^{-t}\dd{t}
	$$

	\resolucion

	$$
		\int_{0}^{\infty} t^{7/2} e^{-t}\dd{t}=\int_{0}^{\infty} t^{5/2-1} e^{-t}\dd{t}=\Gamma \left( \frac{5}{2} \right)
	$$

	Usamos la Formula de Reducción $\Gamma(z+1)=z \Gamma(z)	$, usamos el resultado del ejemplo \ref{ex:ejmplo_01_01},
	$$
		\therefore \int_{0}^{\infty} t^{7/2} e^{-t}\dd{t}=\frac{3}{4}\sqrt{\pi}
	$$
}
\ex{}{	Calcular el siguiente límite para todo $a>1$

	$$
		\lim\limits_{n \to \infty} \frac{\Gamma(n!+1) \left( \sqrt[\Gamma(n!+1)]{a}-1\right) }{\Gamma(n-1)}
	$$

	\resolucion

	$$
		\lim\limits_{n \to \infty} \frac{\Gamma(n!+1) \left( \sqrt[\Gamma(n!)]{a}-1\right) }{\Gamma(n-1)}=	\lim\limits_{n \to \infty} \frac{\Gamma(n!+1) \left( \sqrt[\Gamma(n!)]{a}-1\right) }{n!}
	$$
	$$
		\lim\limits_{n \to \infty} \frac{n!\Gamma(n!) \left( \sqrt[\Gamma(n!)]{a}-1\right) }{n!}=\lim\limits_{n \to \infty}  \Gamma(n!) \left( \sqrt[\Gamma(n!)]{a}-1\right)
	$$

	Hacemos el siguiente cambio de variable $x=\frac{1}{\Gamma(n!)}  $

	$$
		\lim\limits_{x \to 0 ^+} \frac{a^x-1}{x}=\ln(a)
	$$
	$$
		\therefore  \lim\limits_{n \to \infty} \frac{\Gamma(n!+1) \left( \sqrt[\Gamma(n!+1)]{a}-1\right) }{\Gamma(n-1)} = \ln(a)
	$$
}

\mprop{Producto de Weierstrass}{
	La función gamma admite la representación infinita:

	\[
		\frac{1}{\Gamma(z)} = z e^{\gamma z} \prod_{n=1}^\infty \left(1 + \frac{z}{n} \right) e^{-z/n}, \quad z \in \mathbb{C} \setminus \{0, -1, -2, \dots\}
	\]
	Es decir,
	$$
		\Gamma(z)=\dfrac{e^{-\gamma z}}{z} \prod_{n=1}^{\infty} \left(1+\dfrac{z}{n} \right)^{-1} e^{z/n}
	$$


	donde la constante \( \gamma \), llamada \textbf{constante de Euler–Mascheroni}, se define como:

	\[
		\gamma := \lim_{m \to \infty} \left( \sum_{k=1}^m \frac{1}{k} - \ln m \right) \approx 0.5772\ 1566\ 4900\ldots
	\]
}

\mprop{Fórmula de reflexión de Euler}{
	Para todo \( z \notin \mathbb{Z} \), se cumple:
	\[
		\Gamma(z) \Gamma(1 - z) = \frac{\pi}{\sin(\pi z)}.
	\]}

\begin{proof}
	Utilizamos la fórmula del producto de Weierstrass para la función Gamma:
	\[
		\frac{1}{\Gamma(z)} = z e^{\gamma z} \prod_{n=1}^\infty \left(1 + \frac{z}{n} \right) e^{-z/n},
	\]

	Entonces:
	\[
		\Gamma(z) = \frac{1}{z} e^{-\gamma z} \prod_{n=1}^\infty \left(1 + \frac{z}{n} \right)^{-1} e^{z/n}.
	\]

	Análogamente:
	\[
		\Gamma(1 - z) = \frac{1}{1 - z} e^{-\gamma(1 - z)} \prod_{n=1}^\infty \left(1 + \frac{1 - z}{n} \right)^{-1} e^{(1 - z)/n}.
	\]

	Además \( \pi / \sin(\pi z) \), es igual a:
	\[
		\frac{\pi}{\sin(\pi z)} = \frac{1}{z} \prod_{n=1}^\infty \left(1 - \frac{z^2}{n^2} \right)^{-1}.
	\]

	\[
		\Rightarrow \Gamma(z)\Gamma(1 - z) = \frac{\pi}{\sin(\pi z)}.
		\qedhere
	\]
\end{proof}

\ex{$\Gamma(\frac{1}{2}) = \sqrt{\pi}$}{
	Haciendo $z = \frac{1}{2}$ en la reflexión de Euler
	\begin{align*}
		\Gamma\qty(\frac{1}{2}) \Gamma\qty(1-\frac{1}{2}) & = \frac{\pi}{\sin(\pi \cdot \frac{1}{2})} \\
		\pqty{\Gamma\qty(\frac{1}{2})}^2                  & = \pi                                     \\
		\Gamma\qty(\frac{1}{2})                           & = \sqrt{\pi}
	\end{align*}
}

\ex{ Calcule el límite usando la formula de Reflexión de Euler-Función Gamma}{

	$$
		\lim_{x \to  0 }\frac{\sin(x)}{x}
	$$
	\resolucion

	Usamos Reflexión de Euler

	$$
		\lim_{x \to  0 } \frac{1}{x} \cdot \frac{\pi}{\Gamma\left( \ \frac{x}{\pi}\right)  \Gamma\left( 1-\frac{ x}{\pi} \right)  }
	$$

	$$
		\lim_{x \to  0 } \frac{1}{\dfrac{x}{\pi}} \cdot \frac{1}{\Gamma\left( \frac{x}{\pi}\right)  \Gamma\left( 1-\frac{ x}{\pi} \right)  }
	$$

	Hacemos un cambio de variable $x/\pi=z \Rightarrow  z \to 0$

	$$
		\lim_{z \to  0 } \dfrac{1}{z} \cdot \frac{1}{\Gamma\left( z\right)  \Gamma\left( 1-z \right)  }
	$$

	Sabemos que $\Gamma(z+1)=z\Gamma(z)$

	$$
		\lim_{z \to  0 } \cdot \frac{1}{\Gamma\left( z+1\right)  \Gamma\left( 1-z \right)  } =\frac{1}{\Gamma(1+0)\Gamma(1-0)}=1
	$$

	$$
		\therefore \lim_{x \to  0 }\frac{\sin(x)}{x}=1
	$$}
\section{ Otras Propiedades Fundamentales}

\begin{itemize}

	\item \textbf{Fórmula de Stirling (asimptótica):}
	      \[
		      \Gamma(z) \sim \sqrt{2\pi} z^{z - 1/2} e^{-z} \left( 1 + \frac{1}{12z} + \frac{1}{288z^2} - \cdots \right)
	      \]
	\item \textbf{Fórmula de duplicación de Legendre:}
	      \[
		      \Gamma(2z) = \frac{\Gamma(z)\Gamma\left(z + \tfrac{1}{2} \right)}{2^{1 - 2z} \sqrt{\pi}}
	      \]

	      $$
		      2^{2z-1}\Gamma(z)\Gamma\left( z+\dfrac{1}{2}\right) =\sqrt{\pi}\Gamma(2z)
	      $$

	      $$
		      \Gamma\left( n+\dfrac{1}{2}\right) =\dfrac{(2n)!}{4^n n!}\sqrt{\pi} \hspace{1cm} n=1,2,3,\ldots
	      $$

	\item \textbf{Fórmula de triplicación:}

	      \[
		      \Gamma(3z) = \frac{3^{3z - \frac{1}{2}}}{2\pi} \, \Gamma(z) \, \Gamma\left(z + \frac{1}{3} \right) \, \Gamma\left(z + \frac{2}{3} \right)
	      \]

	\item \textbf{Fórmula de multiplicación de Gauss:}

	      Para \( n \in \mathbb{N} \), se tiene:

	      \[
		      \Gamma(nz) = (2\pi)^{\frac{1-n}{2}} \, n^{nz - \frac{1}{2}} \prod_{k=0}^{n-1} \Gamma\left(z + \frac{k}{n} \right)
	      \]
	      $$
		      (2\pi)^{(n-1)/2}  n^{1/2-nz}\Gamma(nz)=\prod_{k=0}^{n-1}\Gamma\left( z+\dfrac{k}{n}\right)
	      $$

	      $$
		      {\displaystyle \Gamma (z)\;\Gamma \left(z+{\frac {1}{k}}\right)\;\Gamma \left(z+{\frac {2}{k}}\right)\cdots \Gamma \left(z+{\frac {k-1}{k}}\right)=(2\pi )^{\frac {k-1}{2}}\;k^{1/2-kz}\;\Gamma (kz)\,\!}
	      $$
	\item \textbf{Representación de Gauss }
	      \[
		      \Gamma(z) = \lim_{n \to \infty} \frac{n! \, n^z}{z (z+1) (z+2) \cdots (z+n)} = \lim_{n \to \infty} \frac{n! \, n^z}{\prod_{k=0}^{n} (z + k)}
		      \quad \text{para } z \notin \{0, -1, -2, \dots\}
	      \]
	\item \textbf{Formula de Knar's  }

	      $$
		      \Gamma(1+z)=2^{2z}\prod_{k=1}^{\infty} \left(\pi^{-1/2} \Gamma\left( \dfrac{1}{2}+2^{-k}z \right)  \right)
	      $$

	\item \textbf{Euler's Formula:}

	      $$
		      \Gamma(z)=\lim\limits_{n \to \infty} \dfrac{n^z}{z}\prod_{k=1}^{n} \dfrac{k}{k+z}=\dfrac{1}{z}\prod_{k=}^{\infty}\dfrac{\left(1+\frac{1}{k} \right)^z }{1+\frac{z}{k}}
	      $$

	\item \textbf{Coeficiente binomial generalizado:}

	      \[
		      \binom{z}{w} = \frac{\Gamma(z+1)}{\Gamma(w+1)\Gamma(z-w+1)}
		      \quad \text{para } z, w \in \mathbb{C} \setminus \mathbb{Z}_{< -1}
	      \]

	\item \textbf{Propiedades en el plano complejo:}

	      \begin{align*}
		      \Gamma(\overline{z})     & = \overline{\Gamma(z)}                                        \\
		      \ln \Gamma(\overline{z}) & = \overline{\ln \Gamma(z)}                                    \\
		      |\Gamma(x + i y)|        & \leq |\Gamma(x)| \quad \text{para } x > 0,\, y \in \mathbb{R} \\
		      \arg \Gamma(z+1)         & = \arg \Gamma(z) + \arg z
	      \end{align*}

	\item \textbf{Serie de potencias de \( \ln \Gamma(1+z) \):}

	      Para \( |z| < 1 \):

	      \[
		      \ln \Gamma(1 + z) = -\gamma z + \sum_{n=2}^\infty \frac{\zeta(n)}{n} (-z)^n
	      \]

	      donde \( \zeta(n) \) es la función zeta de Riemann.

	\item \textbf{Expansión en serie de \( \dfrac{1}{\Gamma(z)} \):}

	      Hay una expansión entera (ver proposición de Weierstrass),  también existe una expansión de tipo:

	      \[
		      \frac{1}{\Gamma(z)} = \sum_{n=1}^\infty c_n z^n, \quad \text{con } c_n \in \mathbb{R}
		      \quad \text{(convergente en } \mathbb{C})
	      \]

	\item \textbf{Aproximación de Stirling's}


	      Para \( |z| \to \infty \), \( |\arg z| < \pi \):

	      \[
		      \Gamma(z) \sim \sqrt{2\pi} \, z^{z - \frac{1}{2}} e^{-z} \left( 1 + \frac{1}{12z} + \frac{1}{288z^2} - \frac{139}{51840z^3} + \cdots \right)
	      \]

	      La serie asintomática para la función Gamma
	      $$
		      \Gamma(n+1) \sim  e^{-n} n^n \sqrt{2 \pi n} \left(1+\dfrac{1}{12n}+\dfrac{1}{288n^2} - \dfrac{139}{51840z^3}-\dfrac{571}{2488320z^4}+\ldots\right)
	      $$

	      Para $n$ suficientemente grande,
	      $$
		      \ln(n!)\sim n\ln(n)-n
	      $$

	      En los problemas vamos a usar con más recuerrencia

	      $$
		      n!\sim \sqrt{2\pi n}\left(\dfrac{n}{e} \right)^n
	      $$

	      \ex{ Calcule el siguiente límite}{\label{ejem1}

		      $$
			      \lim\limits_{n \to \infty} \dfrac{\sqrt[n]{n!}}{n}
		      $$

		      \resolucion

		      Usamos la Aproximación de Stirling $  n!\approx n^n e^{-n}\sqrt{2\pi n }$

		      $$
			      \lim_{n\to \infty}  \dfrac{\sqrt[n]{n^n e^{-n}\sqrt{2\pi n} }}{n}=	\lim_{n\to \infty}  \dfrac{ n e^{-1}\sqrt[2n]{2\pi n }}{n}
		      $$

		      Se sabe que $\lim_{n\to\infty} \sqrt[2n]{2\pi n}=1$

		      $$
			      \therefore 	\lim\limits_{n \to \infty} \dfrac{\sqrt[n]{n!}}{n}=\dfrac{1}{e}
		      $$
	      }

	      \ex{Encuentre el valor del límite si  $k=1,2,3,\ldots$ }{
		      $$
			      \lim\limits_{n \to \infty} \dfrac{\ln(n^n)}{\ln((kn)!)}
		      $$
		      \resolucion
		      Usamos la Aproximación de Stirling $  n!\sim n^n e^{-n}\sqrt{2\pi n }$

		      $$
			      \lim\limits_{n \to \infty} \dfrac{\ln(n^n)}{\ln((kn)^{(kn)} e^{-kn}\sqrt{2\pi kn })}
		      $$

		      Sabemos que $\ln(ab)=\ln(a)+\ln(b) ; a,b >0$

		      $$
			      \lim\limits_{n \to \infty} \dfrac{n\ln(n)}{kn\ln(kn/e) +\ln(\sqrt{2\pi kn })}
		      $$

		      Dividimos entre $n$ al numerador como denominador

		      $$
			      \lim\limits_{n \to \infty} \dfrac{\ln(n)}{k\ln(kn/e) + \ln(\sqrt[2n]{2\pi kn })}
		      $$

		      Es facil comprobar que $\lim_{n\to\infty} \ln( \sqrt[2n]{2\pi k n} ) =0$, además $k/e$ es una constante $c$

		      $$
			      \lim\limits_{n \to \infty} \dfrac{\ln(n)}{k\ln(cn)}=\dfrac{1}{k}
		      $$

		      $$
			      \therefore \lim\limits_{n \to \infty} \dfrac{\ln(n^n)}{\ln((kn)!)}=\dfrac{1}{k}
		      $$
	      }


	      \ex{Encuentre el valor del límite si  $k=1,2,3,\ldots$}{
		      $$
			      \lim\limits_{n \to \infty} \dfrac{\ln(n^n)}{\ln((kn)!)}
		      $$

		      \resolucion

		      {\bf Método 1}

		      Usamos la Aproximación de Stirling $  n!\sim n^n e^{-n}\sqrt{2\pi n }$

		      $$
			      \lim\limits_{n \to \infty} \dfrac{\ln(n^n)}{\ln((kn)^{(kn)} e^{-kn}\sqrt{2\pi kn })}
		      $$

		      Sabemos que $\ln(ab)=\ln(a)+\ln(b) ; a,b >0$

		      $$
			      \lim\limits_{n \to \infty} \dfrac{n\ln(n)}{kn\ln(kn/e) +\ln(\sqrt{2\pi kn })}
		      $$

		      Dividimos entre $n$ al numerador como denominador

		      $$
			      \lim\limits_{n \to \infty} \dfrac{\ln(n)}{k\ln(kn/e) + \ln(\sqrt[2n]{2\pi kn })}
		      $$

		      Es facil comprobar que $\lim_{n\to\infty} \ln( \sqrt[2n]{2\pi k n} ) =0$, además $k/e$ es una constante $c$

		      $$
			      \lim\limits_{n \to \infty} \dfrac{\ln(n)}{k\ln(cn)}=\dfrac{1}{k}
		      $$

		      $$
			      \therefore \lim\limits_{n \to \infty} \dfrac{\ln(n^n)}{\ln((kn)!)}=\dfrac{1}{k}
		      $$

	      }


	      \ex{}{ \label{ejem2}   Encuentre el valor del límite para todo $k=1,2,3,\ldots $

		      $$
			      \lim_{n \to \infty} \left( \frac{(kn)!}{(k!n^k )^n  }\right)^{1/kn}
		      $$
		      \resolucion

		      Aproximación de Stirling $  t!\sim t^t e^{-t}\sqrt{2\pi t}$ si $ t \to \infty $

		      $$
			      \lim_{n \to \infty} \left( \frac{ (kn)^{kn} e^{-kn} \ \sqrt{2\pi kn} }{  (k! n^k)^n  }\right)^{1/kn}
		      $$

		      $$
			      \lim_{n \to \infty}  \frac{ (kn) e^{-1} \sqrt[2kn]{2\pi kn} }{  \sqrt[k]{k!} n }
		      $$

		      En los ejemplos anteriores, vimos que  $\lim_{n\to\infty}  \sqrt[2nk]{2\pi k n} =1$
		      $$
			      \lim_{n \to \infty}  \dfrac{ k \sqrt[2kn]{2\pi kn} }{ e \sqrt[k]{k!}  } = \frac{ k}{ e \sqrt[k]{k!}  }
		      $$


		      $$
			      \therefore \lim_{n \to \infty} \left( \frac{(kn)!}{(k!n^k )^n  }\right)^{1/kn} =\frac{ k}{ e \sqrt[k]{k!}  }
		      $$

	      }

	      \ex{Calcule el siguiene límite $ k, n >0$}{
	      $$
		      \lim_{x \to \infty} \left(\frac{(x+k)^{x^{n+1}} \sqrt{x^{2x^n-x^{n-1}} }    }{x^{x^{n+1}} \pqty{\Gamma(x)}^{x^{n-1}} } \right) ^{x^{-n}}
	      $$

	      \resolucion

	      Sea $L$ el límite a calcular, reconocemos $(ab)^n=a^n \cdot b^n$
	      $$
		      L=  \lim_{x \to \infty}\left(\dfrac{(x+k)^{x^{n+1}}     }{x^{x^{n+1}} } \right) ^{x^{-n}}\cdot
		      \left(\dfrac{ \sqrt{x^{2x^n-x^{n-1}} }    }{ \pqty{\Gamma(x)}^{x^{n-1}} } \right) ^{x^{-n}}
	      $$

	      Tambien vemos $a^n \cdot a^m=a^{n+m}$

	      $$
		      L=  \lim_{x \to \infty}\dfrac{(x+k)^{x^{n+1-n}}     }{x^{x^{n+1-n}} } \cdot
		      \lim_{x \to \infty} \left(\dfrac{ \sqrt{x^{2x^{n-n+1}-x^{n-1-n+1}} }    }{ \pqty{\Gamma(x)}^{x^{n-1-n+1}} } \right)^{x^{-1}}
	      $$

	      $$
		      L= \lim_{x \to \infty}\dfrac{(x+k)^x     }{x^x } \cdot
		      \lim_{x \to \infty} \left(\dfrac{ \sqrt{x^{2x-1} }    }{ \Gamma(x) } \right)^{x^{-1}}
	      $$

	      Relacionamos la función Gamma con el factorial $\Gamma(x)=(x-1)!$ \ \ y \ \     $ x(x-1)!=x! $

	      $$
		      L=  \lim_{x \to \infty} \left( \dfrac{x+k    }{x }\right)^x  \cdot
		      \lim_{x \to \infty} \left(\dfrac{x^{x-1/2} }{ (x-1)! } \right)^{x^{-1}}
		      =
		      e^k \cdot
		      \lim_{x \to \infty} \left(\dfrac{x^{x+1/2} }{ x! } \right)^{x^{-1}}
	      $$

	      Aproximación de Stirling $  n!\sim n^n e^{-n}\sqrt{2\pi n }$

	      $$
		      L=   e^k\lim_{x \to \infty} \left(\dfrac{x^{x+1/2} }{\sqrt{2\pi}\  x^{x+1/2} \ e^{-x}} \right)^{x^{-1}}
		      =
		      e^k\lim_{x \to \infty} \left(\dfrac{e^x }{\sqrt{2\pi}  } \right)^{x^{-1}}
	      $$

	      Facilmente se comprueba que  $\lim_{x\to\infty} \sqrt[2x]{2\pi }=1$, entonces
	      $$
		      L= e^k\lim_{x \to \infty} \dfrac{e}{\sqrt[2x]{2\pi}  } =e^k\cdot e=e^{k+1}
	      $$
	      Finalmente, tenemos el valor de $L$,
	      $$
		      \therefore  \lim_{x \to \infty} \left(\dfrac{(x+k)^{x^{n+1}} \sqrt{x^{2x^n-x^{n-1}} }    }{x^{x^{n+1}} \Gamma^{x^{n-1}}(x) } \right) ^{x^{-n}}=e^{k+1}
	      $$
	      }

	\item \textbf{Desarrollo asintótico de \( \ln \Gamma(z) \):}

	      \[
		      \ln \Gamma(z) \sim z \ln z - z + \frac{1}{2} \ln(2\pi) + \sum_{n=1}^\infty \frac{B_{2n}}{2n(2n-1) z^{2n-1}}
		      \quad \text{para } |z| \to \infty,\, |\arg z| < \pi
	      \]

	      donde \( B_{2n} \) son los números de Bernoulli.

	\item \textbf{Expansión de Laurent}

	      $$
		      \Gamma(z)=\frac{1}{z}-\gamma +\frac{(\gamma^2+\zeta(2))z }{2}+\mathcal{O}(z^2)
	      $$

	      En los siguientes capitulos estudiaremos las siguientes constantes:

	      La constante de Euler Mascheroni $\gamma \approx 0.577 \ldots $

	      La función Zeta de Riemman de dos $\zeta(2)=\frac{\pi^2}{6}$

	      \ex{Calcular el siguiente límite}{
		      $$
			      \lim\limits_{t \to 0} \frac{\Gamma(t+1)+\gamma t-1}{t^2}
		      $$

		      donde $\gamma$ es la constante de Euler-Mascheroni

		      \resolucion

		      Por la expansión de Laurent
		      $$
			      \Gamma(t+1)=1-\gamma t + \frac{(\gamma^2+\zeta(2))t^2 }{2}+\mathcal{O}(t^3)
		      $$

		      $$
			      \Gamma(t+1)+\gamma t-1=  \frac{(\gamma^2+\zeta(2))t^2 }{2}+\mathcal{O}(t^3)
		      $$

		      $$
			      \frac{\Gamma(t+1)+\gamma t-1}{t^2}= \frac{(\gamma^2+\zeta(2))}{2}+\mathcal{O}(t)
		      $$

		      $$
			      \lim\limits_{t \to 0 }	\frac{\Gamma(t+1)+\gamma t-1}{t^2}=\lim\limits_{t \to 0 } \left(   \frac{(\gamma^2+\zeta(2))}{2}+\mathcal{O}(t)\right)
		      $$


		      $$
			      \lim\limits_{t \to 0 }	\frac{\Gamma(t+1)+\gamma t-1}{t^2}=   \frac{(\gamma^2+\zeta(2))}{2}+\cancelto{0}{\lim\limits_{t \to 0 } \mathcal{O}(t)}
		      $$

		      La función Zeta de Riemman de dos es,  $\zeta(2)=\frac{\pi^2}{6}$

		      $$
			      \therefore \lim\limits_{t \to 0 }	\frac{\Gamma(t+1)+\gamma t-1}{t^2}=\frac{\gamma^2}{2}+\frac{\pi^2}{12}
		      $$
	      }
\end{itemize}

\section{Ejercicios }

\subsection{Ejercicios Resueltos}
\begin{enumerate}
	\item Evalue:
	      \[
		      \Gamma\left(\frac{1+i}{2}\right) \Gamma\left(\frac{1-i}{2}\right)
	      \]
	      Solución:
	      \begin{align*}
		      \Gamma\left(\frac{1+i}{2}\right) \Gamma\left(\frac{1-i}{2}\right)
		       & = \Gamma\left(\frac{1+i}{2}\right) \Gamma\left(1 - \frac{1+i}{2}\right)                                                         \\
		       & = \frac{\pi}{\sin\left(\pi \cdot \pqty{\frac{1+i}{2} } \right)} = \frac{\pi}{\sin\left(\frac{\pi}{2} + \frac{\pi i}{2} \right)} \\
		       & = \frac{\pi}{\cos(\frac{\pi i}{2} )} =\frac{\pi}{\cosh(\frac{\pi }{2} )}                                                        \\
		       & =   \pi  \sech(\frac{\pi}{2})
	      \end{align*}
	\item  Cacule,
	      $$
		      \int_a^{a+1} \ln(\Gamma(x)) \dd{x}
	      $$
	      Sol:

	      Sea $f(a)$ la integral,
	      $$
		      f(a)=\int_a^{a+1} \ln(\Gamma(x)) \dd{x}
	      $$
	      diferenciando ambos lados
	      $$
		      f'(a) = \ln(\Gamma(a+1)) -\ln(\Gamma(a)) = \ln(a)
	      $$
	      Integrando respecto a $a$ en ambos lados,
	      $$
		      f(a)= a\ln(a)-a+C
	      $$
	      si $a \to 0^{+}$, tenemos que
	      $$
		      C = \int_0^1 \ln(\Gamma(x)) \dd{x} = \ln(\sqrt{2\pi})
	      $$
	      Finalmente,
	      $$
		      \int_a^{a+1} \ln(\Gamma(x)) \dd{x} = a\ln(a)-a+\ln(\sqrt{2\pi}) = \ln(\sqrt{2\pi} \pqty{\frac{a}{e}}^a)
	      $$
	\item 	$$
		      \lim_{k \to \infty}  \ \lim_{n \to \infty} \left( \frac{(kn)!}{(k!n^k )^n  }\right)^{1/kn}
	      $$

	      \resolucion

	      Por el Ejemplo \ref{ejem2}

	      $$
		      \lim_{k \to \infty}\left[   \ \lim_{n \to \infty} \left( \dfrac{(kn)!}{(k!n^k )^n  }\right)^{1/kn}\right]  = \lim_{k \to \infty} \left[ \dfrac{ k}{ e \sqrt[k]{k!}  } \right]
	      $$

	      \noindent

	      Por el Ejemplo \ref{ejem1}

	      $$
		      \lim_{k \to \infty} \left[ \dfrac{ k}{ e \sqrt[k]{k!}  } \right] =\dfrac{1}{e \left( \dfrac{1}{e} \right) }=1
	      $$

	      $$
		      \therefore \lim_{k \to \infty}  \ \lim_{n \to \infty} \left( \frac{(kn)!}{(k!n^k )^n  }\right)^{1/kn} =1
	      $$
	\item
\end{enumerate}

\subsection{Ejercicios Propuestos}

\begin{enumerate}

	\item Evaluar la siguiente integral,
	      \[
		      \int_0^\infty x^n e^{-\lambda x} \, dx
		      \quad \textbf{Respuesta: } \frac{\Gamma(n+1)}{\lambda^{n+1}}
	      \]

	\item Evaluar:
	      \[
		      \int_0^1 (-\ln t)^s \, \dd{t}
		      \quad \textbf{Respuesta: } \Gamma(s+1)
	      \]

	\item Demostrar que:
	      \[
		      \int_0^\infty e^{-x^2} \, dx = \frac{\sqrt{\pi}}{2}
		      \quad \text{usando } \Gamma\left(\frac{1}{2}\right)
	      \]

	\item Mostrar que:
	      \[
		      \int_0^\infty \frac{t^{z-1}}{1 + t} \, \dd{t} = \pi / \sin(\pi z)
		      \quad \text{para } 0 < \operatorname{Re}(z) < 1
	      \]
	\item
	      Encuentre el valor del límite si $a \in \mathbb{Z^{+}}$

	      $$
		      \lim_{k \to \infty}    \left(\frac{e^{(k!)^{-a} +(k!)^k}-k^{(k^2)!}}{\left( (ak)!\right) ^{ak}} \right)
	      $$

\end{enumerate}

\subsection{Gamma Incompleta}

\dfn{Función Gamma Incompleta}{
	$$
		\Gamma(z,x)=\int_x^{\infty} t^{z-1 } e^{-t} \dd{t}  \hspace{1cm} \Re(z)>0
	$$
}

\dfn{Función gamma Incompleta- Función Gamma Inferior}{
	$$
		\gamma(z,x)=\int_0^{x} t^{z-1 } e^{-t} \dd{t}  \hspace{1cm} \Re(z)>0
	$$
}

\dfn{Simbolo de Kramp}{
	$$
		c^{a/b}=c(c+b)(c+2b)\ldots\left[ c+(a-1)b \right] =\dfrac{b^a  \Gamma\left(a+ \dfrac{c}{b}\right)}{\Gamma\left( \dfrac{c}{b}\right) }
	$$ }

\thm{}{

	Con la representaciónde la función Gamma, podemos extender  a 	$x > -1$.

	$$
		\bigg\rvert  \int_0^{\infty} e^{-t}t^x\dd{t} \bigg\rvert  \leq  \int_{\epsilon}^{\infty} e^{-t}t^x\dd{t} <\infty
	$$

	Cuando cercamos a cero $x=-z$ se tiene

	$$
		\bigg\rvert  \int_0^{\epsilon} \dfrac{e^{-t}}{t^z }  \dd{t} \bigg\rvert  \sim   \int_0^{\epsilon} \dfrac{1}{t^z}\dd{t} <\infty
	$$
}

\subsubsection{Relación entre la Función Gamma Superior e Inferior}

\[ \Gamma(z)=\gamma(z,x)+\Gamma(z,x) \left\{ \begin{array}{rcl}
		 & \Gamma(z+1,x)=z\Gamma(z,x)+x^z e^{-x} \\
		 &                                       \\
		 & \gamma(z+1,x)=z\gamma(z,x)-x^z e^{-x} \\
	\end{array}
	\right. \]

$$
	\Gamma(z)=\lim_{x\to\infty} \gamma(z,x)
$$



\chapter{Función Beta}
\section{Definición}

\dfn{definition}{
	La \textbf{función beta} o \textbf{función de Euler} se define para \( x > 0 \) y \( y > 0 \) como
	\[
		B(x, y) = \int_0^1 t^{x-1} (1 - t)^{y-1} \, dt.
	\]
}
Esta integral converge para todos \( x, y > 0 \). Es simétrica:
\[
	B(x, y) = B(y, x).
\]

\section{Propiedades}

\begin{itemize}
	\item \textbf{Simetría:} \( B(x, y) = B(y, x) \)
	      \begin{myproof}
		      Recordemos la definición de la función beta:
		      \[
			      B(x, y) = \int_0^1 t^{x-1}(1 - t)^{y - 1} \, dt.
		      \]
		      Realizamos el cambio de variable \( u = 1 - t \), entonces \( t = 1 - u \), y \( dt = -du \). Los límites cambian:
		      \[
			      t = 0 \Rightarrow u = 1, \quad t = 1 \Rightarrow u = 0.
		      \]
		      Invirtiendo los límites por el signo de \( du \), tenemos:
		      \begin{align*}
			      B(x, y) & = \int_0^1 t^{x-1}(1 - t)^{y - 1} \, dt                                                 \\
			              & = \int_1^0 (1 - u)^{x-1} u^{y-1} (-du) = \int_0^1 u^{y-1}(1 - u)^{x-1} \, du = B(y, x).
		      \end{align*}
		      Por tanto, se concluye que \( B(x, y) = B(y, x) \).
	      \end{myproof}


	\item
	      La función beta se relaciona con la función gamma mediante la fórmula:
	      \[
		      B(x, y) = \frac{\Gamma(x)\Gamma(y)}{\Gamma(x + y)}.
	      \]


	      \begin{myproof}
		      Recordemos la definición de la función gamma:
		      \[
			      \Gamma(z) = \int_0^\infty t^{z-1} e^{-t} \, dt, \quad \text{para } \operatorname{Re}(z) > 0.
		      \]
		      Sea \( f(t) = t^{x-1} e^{-t} \), \( g(t) = t^{y-1} e^{-t} \). Estas funciones son de tipo exponencial, por lo tanto sus transformadas de Laplace existen y son:
		      \[
			      \mathcal{L}\{f\}(s) = \Gamma(x)s^{-x}, \quad \mathcal{L}\{g\}(s) = \Gamma(y)s^{-y}.
		      \]
		      Por la propiedad de convolución de la transformada de Laplace:
		      \[
			      \mathcal{L}\{f * g\}(s) = \mathcal{L}\{f\}(s) \cdot \mathcal{L}\{g\}(s) = \Gamma(x)\Gamma(y)s^{-x-y}.
		      \]

		      Ahora calculamos la convolución \( (f * g)(t) \):
		      \[
			      (f * g)(t) = \int_0^t \tau^{x-1} (t - \tau)^{y-1} e^{-\tau} e^{-(t - \tau)} \, d\tau = e^{-t} \int_0^t \tau^{x-1} (t - \tau)^{y-1} \, d\tau.
		      \]

		      Usamos el cambio de variable \( \tau = tu \), \( d\tau = t\,du \), lo que lleva a:
		      \[
			      (f * g)(t) = e^{-t} \int_0^1 (tu)^{x-1} (t - tu)^{y-1} t\,du = e^{-t} t^{x + y - 1} \int_0^1 u^{x-1}(1 - u)^{y-1} du.
		      \]

		      La integral que aparece es precisamente la función beta:
		      \[
			      \int_0^1 u^{x-1}(1 - u)^{y-1} du = B(x, y).
		      \]
		      Por tanto, obtenemos:
		      \[
			      (f * g)(t) = e^{-t} t^{x + y - 1} B(x, y).
		      \]

		      Apliquemos ahora la transformada de Laplace de \( f * g \):
		      \[
			      \mathcal{L}\{f * g\}(s) = \int_0^\infty e^{-st} e^{-t} t^{x + y - 1} B(x, y)\,dt = B(x, y) \int_0^\infty t^{x + y - 1} e^{-(s+1)t} dt.
		      \]

		      Esta última integral es:
		      \[
			      \int_0^\infty t^{x + y - 1} e^{-(s+1)t} dt = \Gamma(x + y)(s + 1)^{-(x + y)}.
		      \]

		      Entonces:
		      \[
			      \mathcal{L}\{f * g\}(s) = B(x, y)\Gamma(x + y)(s + 1)^{-(x + y)}.
		      \]

		      Por otro lado, recordamos que:
		      \[
			      \mathcal{L}\{f * g\}(s) = \Gamma(x)\Gamma(y)s^{-(x + y)}.
		      \]

		      Comparando las dos expresiones de \( \mathcal{L}\{f * g\}(s) \) con el cambio \( s \to s + 1 \), se concluye que:
		      \[
			      \Gamma(x)\Gamma(y) = B(x, y)\Gamma(x + y) \quad \Rightarrow \quad B(x, y) = \frac{\Gamma(x)\Gamma(y)}{\Gamma(x + y)}.
		      \]
	      \end{myproof}

	\item \textbf{Para enteros positivos:} Si \( x, y \in \mathbb{N} \), entonces:
	      \[
		      B(x, y) = \frac{(x-1)!(y-1)!}{(x + y - 1)!}
	      \]
	      \begin{myproof}
		      Si \( x, y \in \mathbb{N} \), entonces por definición de factorial en términos de Gamma:
		      \[
			      \Gamma(n) = (n-1)! \quad \text{para } n \in \mathbb{N}.
		      \]
		      Aplicando la fórmula \( B(x,y) = \dfrac{\Gamma(x)\Gamma(y)}{\Gamma(x+y)} \), obtenemos:
		      \[
			      B(x, y) = \frac{(x-1)!(y-1)!}{(x + y - 1)!}.
		      \]
	      \end{myproof}
	\item \textbf{Propiedad de recurrencia:}
	      \[
		      B(x+1, y) = \frac{x}{x + y} B(x, y)
	      \]
	      \begin{myproof}
		      Recordamos la relación con la función gamma:
		      \[
			      B(x, y) = \frac{\Gamma(x)\Gamma(y)}{\Gamma(x + y)}.
		      \]
		      Entonces:
		      \[
			      B(x+1, y) = \frac{\Gamma(x+1)\Gamma(y)}{\Gamma(x + y + 1)}.
		      \]
		      Usamos la propiedad funcional de la gamma: \( \Gamma(z+1) = z\Gamma(z) \), aplicada a \( x+1 \) y \( x + y + 1 \):
		      \[
			      \Gamma(x+1) = x\Gamma(x), \quad \Gamma(x + y + 1) = (x + y)\Gamma(x + y).
		      \]
		      Por tanto:
		      \[
			      B(x+1, y) = \frac{x \Gamma(x)\Gamma(y)}{(x + y)\Gamma(x + y)} = \frac{x}{x + y} \cdot \frac{\Gamma(x)\Gamma(y)}{\Gamma(x + y)} = \frac{x}{x + y} B(x, y).
		      \]
	      \end{myproof}

\end{itemize}

\section{Ejemplos Resueltos}

\begin{example}
	Calcular \( B(3,2) \).

	Usamos la relación con la función gamma:
	\[
		B(3,2) = \frac{\Gamma(3)\Gamma(2)}{\Gamma(5)} = \frac{2! \cdot 1!}{4!} = \frac{2 \cdot 1}{24} = \frac{1}{12}.
	\]
\end{example}

\begin{example}
	Calcular \( B\left(\frac{1}{2}, \frac{1}{2}\right) \).

	Sabemos que \( \Gamma(1/2) = \sqrt{\pi} \), entonces:
	\[
		B\left(\frac{1}{2}, \frac{1}{2}\right) = \frac{\Gamma(1/2)^2}{\Gamma(1)} = \frac{\pi}{1} = \pi.
	\]
\end{example}

\section{Propiedades Avanzados}
\mprop{}{
	\[
		B(x, y) = \int_0^\infty \frac{t^{x-1}}{(1 + t)^{x+y}} \dd{t}
	\]
}
\begin{myproof}
	Partimos de la definición de la función beta:
	\[
		B(x, y) = \int_0^1 t^{x-1}(1 - t)^{y-1} \dd{t}, \quad x, y > 0.
	\]
	Hacemos el cambio de variable \( t = \frac{u}{1+u} \), con \( u \in (0, \infty) \). Entonces:
	\[
		\dd{t} = \frac{1}{(1+u)^2} \dd{u}, \quad 1 - t = \frac{1}{1+u}.
	\]
	Además:
	\[
		t^{x-1} = \left( \frac{u}{1+u} \right)^{x-1}, \quad (1 - t)^{y-1} = \left( \frac{1}{1+u} \right)^{y-1}.
	\]
	Por lo tanto:
	\begin{align*}
		B(x, y) & = \int_0^\infty \left( \frac{u}{1+u} \right)^{x-1} \left( \frac{1}{1+u} \right)^{y-1} \cdot \frac{1}{(1+u)^2} \dd{u} \\
		        & = \int_0^\infty \frac{u^{x-1}}{(1+u)^{x + y}} \dd{u}.
	\end{align*}
	Así se obtiene la identidad deseada.
\end{myproof}

\mprop{}{
	\[
		B(x, y) = 2 \int_0^{\pi/2} \cos^{2x-1}(t) \sin^{2y-1}(t) \dd{t}
	\]
}
\begin{myproof}
	Partimos nuevamente de la definición:
	\[
		B(x, y) = \int_0^1 t^{x-1}(1 - t)^{y-1} \dd{t}.
	\]
	Usamos el cambio de variable \( t = \sin^2 \theta \), con \( \theta \in (0, \pi/2) \). Entonces:
	\[
		\dd{t} = 2\sin\theta\cos\theta \dd{\theta}, \quad 1 - t = \cos^2\theta.
	\]
	Además:
	\[
		t^{x-1} = \sin^{2x - 2} \theta, \quad (1 - t)^{y-1} = \cos^{2y - 2} \theta.
	\]
	Sustituyendo:
	\begin{align*}
		B(x, y) & = \int_0^{\pi/2} \sin^{2x - 2}(\theta) \cos^{2y - 2}(\theta) \cdot 2\sin(\theta)\cos(\theta) \dd{\theta} \\
		        & = 2 \int_0^{\pi/2} \sin^{2x - 1}(\theta) \cos^{2y - 1}(\theta) \dd{\theta}.
	\end{align*}
	Así se obtiene la identidad.
\end{myproof}

\subsection{Ejercicios resueltos}
\ex{	Verifique que para \( m, n \in \mathbb{N} \):}{
	\[
		B(m,n) = \left( \frac{1}{m} + \frac{1}{n} \right) \binom{m+n}{m}^{-1}
	\]
	\resolucion
	Recordamos que para \( m, n \in \mathbb{N} \), se tiene:
	\[
		B(m, n) = \frac{(m-1)!(n-1)!}{(m+n-1)!}.
	\]
	También recordamos que el coeficiente binomial se expresa como:
	\[
		\binom{m+n}{m} = \frac{(m+n)!}{m! \, n!}.
	\]
	Entonces:
	\begin{align*}
		\left( \frac{1}{m} + \frac{1}{n} \right) \binom{m+n}{m}^{-1}
		 & = \left( \frac{1}{m} + \frac{1}{n} \right) \cdot \frac{m! \, n!}{(m+n)!} \\
		 & = \left( \frac{n + m}{mn} \right) \cdot \frac{m! \, n!}{(m+n)!}          \\
		 & = \frac{(m+n) \cdot m! \cdot n!}{m n (m+n)!}.
	\end{align*}
	Ahora simplificamos la expresión de \( B(m, n) \):
	\[
		B(m, n) = \frac{(m-1)!(n-1)!}{(m+n-1)!} = \frac{m! \cdot n!}{m \cdot n \cdot (m+n-1)!}.
	\]
	Y como \( (m+n)! = (m+n)(m+n-1)! \), entonces:
	\[
		\frac{(m+n) \cdot m! \cdot n!}{m n (m+n)!} = \frac{m! \cdot n!}{m n (m+n-1)!} = B(m, n).
	\]
	Por tanto, se verifica la identidad:
	\[
		B(m,n) = \left( \frac{1}{m} + \frac{1}{n} \right) \binom{m+n}{m}^{-1}.
	\]
}

\subsection{Ejercicios Propuestos}

\begin{enumerate}

	\item Pruebe la siguiente integral:
	      \[
		      \int_0^\infty \frac{x^{u+1}}{(1+x^2)^2} \, \dd{x}= \frac{u\pi}{4\sin\left(\frac{\pi u}{2}\right)}
	      \]

	\item Calcular:
	      \[
		      \int_0^\infty \frac{\, \dd{x}}{(x^n + 1)^{m+1}}, \quad n \in \mathbb{N},\; m > -1.
	      \]
	      (Sugerencia: usar cambio de variable y función beta.)

	\item Evaluar la siguiente integral con raíz cuártica:
	      \[
		      \int_0^1 \frac{x^n}{\sqrt{1 - x^4}} \, \dd{x}, \quad n \in \mathbb{N}.
	      \]
	      (Sugerencia: considerar cambio \( x^4 = t \).)

	\item Verificar la identidad:
	      \[
		      \int_0^\infty e^{-(k + k^2/x)} \frac{\, \dd{x}}{\sqrt{x}} = \sqrt{\pi} e^{-2k}, \quad k > 0.
	      \]
	      (Sugerencia: cambiar variable a \( x = k t \).)

	\item Sea \( i \) la unidad imaginaria. Evaluar la siguiente integral de valor complejo:
	      \[
		      \int_0^{\frac{\pi}{2}} \tan^i(x) \, \dd{x}.
	      \]

	\item Calcular:
	      \[
		      \int_{-\infty}^{\infty} \left(1+\frac{x^2}{2025}\right)^{-1013} \dd{x}
	      \]
	      (Sugerencia: usar identidad con la función beta.)

	\item Evaluar el siguiente límite en términos de la función beta:
	      \[
		      \lim_{\varepsilon \to 0^+} \int_\varepsilon^{1-\varepsilon} x^{p-1}(1-x)^{q-1} \dd{x}, \quad \text{para } p, q > 0.
	      \]



	\item Sea \( a > 0 \). Demuestre que:
	      \[
		      \int_0^a \frac{x^{p-1}}{(1 - \frac{x}{a})^{1-p}} \dd{x} = a^p B(p, 1 - p), \quad 0 < p < 1.
	      \]

	\item Evaluar la integral:
	      \[
		      \int_0^\pi \sin^{2a - 1}(x) \cos^{2b - 1}(x) \dd{x}, \quad a, b > 0.
	      \]
	      (Sugerencia: cambiar a mitad de período y conectar con la función beta.)

	\item Verifique que:
	      \[
		      \int_0^1 x^{m-1} \ln(1 - x) \dd{x} = -\frac{1}{m^2}, \quad m \in \mathbb{N}.
	      \]
	      (Usar representación en serie o derivar respecto a un parámetro en \( B(x,y) \).)

	\item Sea \( f(a) := \int_0^1 \frac{x^a - 1}{\log x} \dd{x} \). Mostrar que:
	      \[
		      f(a) = \log B(a + 1, 1) = \log\left( \frac{1}{a + 1} \right).
	      \]

	\item Evaluar:
	      \[
		      \int_0^1 x^p (1 - x)^q \log x \dd{x}, \quad p, q > 0.
	      \]
	      (Sugerencia: derivar \( B(p+1, q+1) \) respecto a \( p \).)




	\item Sea \( \alpha > 0 \). Evaluar:
	      \[
		      \int_0^\infty \frac{x^{\alpha - 1}}{(1 + x^2)^{\alpha + \frac{1}{2}}} \dd{x}.
	      \]
	      (Sugerencia: usar la identidad \( B(x,y) = 2\int_0^{\pi/2} \sin^{2x-1}(\theta) \cos^{2y-1}(\theta) \dd{\theta} \).)
	\item Evalúe si es correcto la igualdad,
	      $$
		      \lim\limits_{x \to 1^{-}} \sqrt{1-x} (1+x+x^4+x^9+x^{16}+x^{25}+\cdots) = \frac{\sqrt{\pi}}{2}
	      $$
	\item Evalúe si es correcto la igualdad
	      $$
		      \sum_{a=0}^{n} \binom{n}{a} B(a+1, n-a+1) = 1
	      $$
	\item Calcule y restringe $s$,
	      $$
		      \int_{0}^{2} x^{2s} (4-x^2)^{-1//2} \dd{x}
	      $$
	\item Evalúe,
	      $$
		      \int_{0}^{\pi/(2a)} \tan(ax) \dd{x}
	      $$

	\item
	      $$
		      \sum_{k=0}^{n} \binom{n}{k} \frac{(-1)^{n+k}}{2n+1-k} = \frac{1}{(2n+1) \binom{2n}{n}}
	      $$

\end{enumerate}

\section{Función Beta incompleta}

\dfn{}{

	Sea \( a > 0 \), \( b > 0 \) y \( x \in [0,1] \). La \textbf{función Beta incompleta} se define como:
	\[
		B_x(a,b) = \int_0^x t^{a-1}(1 - t)^{b-1} \, dt
	\]
}

\subsection{ Propiedades}

\begin{itemize}
	\item \textbf{Relación con la función Beta:}
	      \[
		      B_x(a,b) + B_{1-x}(b, a) = B(a,b)
	      \]

	\item \textbf{Relación con funciones hipergeométricas:}
	      \[
		      B_x(a,b) = \frac{x^a}{a} \, {}_2F_1(a, 1 - b; a+1; x)
	      \]

	\item \textbf{Relación con la función Gamma:}
	      \[
		      B(a,b) = \frac{\Gamma(a)\Gamma(b)}{\Gamma(a + b)}
		      \quad \text{y} \quad
		      B_x(a,b) = \int_0^x t^{a-1}(1 - t)^{b-1} dt
	      \]

	\item \textbf{Derivada respecto a \( x \):}
	      \[
		      \frac{d}{dx} B_x(a,b) = x^{a-1}(1 - x)^{b-1}
		      \quad \Rightarrow \quad
		      \frac{d}{dx} I_x(a,b) = \frac{x^{a-1}(1 - x)^{b-1}}{B(a,b)}
	      \]
\end{itemize}

\subsection{  Ejemplos}

\ex{}{Calcular \( B_{1/2}(2,3) \)}

\begin{solution}
	\[
		B_x(a,b) = \int_0^{1/2} t^{2-1}(1 - t)^{3-1} dt = \int_0^{1/2} t(1 - t)^2 dt
	\]
	Expandimos:
	\[
		(1 - t)^2 = 1 - 2t + t^2 \Rightarrow t(1 - t)^2 = t - 2t^2 + t^3
	\]
	\[
		\int_0^{1/2} (t - 2t^2 + t^3) dt = \left[\frac{t^2}{2} - \frac{2t^3}{3} + \frac{t^4}{4} \right]_0^{1/2}
	\]
	\[
		= \frac{1}{8} - \frac{2}{24} + \frac{1}{64} = \frac{1}{8} - \frac{1}{12} + \frac{1}{64} = \frac{5}{192}
	\]
\end{solution}

\subsection{  Ejercicios propuestos}

\begin{enumerate}

	\item  Demuestre que:
	      \[
		      B_x(a,b) + B_{1-x}(b,a) = B(a,b)
	      \]

	\item 	Encuentre \( B_{3/4}(1,2) \) y \( I_{3/4}(1,2) \).
	\item 	Pruebe que:
	      \[
		      \frac{d}{dx} I_x(a,b) = \frac{1}{B(a,b)} x^{a-1}(1 - x)^{b-1}
	      \]
	\item 	Usando la relación con la función hipergeométrica, calcule \( B_x(a,b) \) para \( a = 1 \), \( b = 2 \).
	\item
	      Evalúe el límite:
	      \[
		      \lim_{x \to 1^{-}} I_x(a,b)
		      \quad \text{y} \quad
		      \lim_{x \to 0^{+}} I_x(a,b)
	      \]
	\item
	      Pruebe que si \( a, b > 0 \), entonces \( I_x(a,b) \) es estrictamente creciente en \( x \in (0,1) \).
\end{enumerate}


\chapter{Función Digamma}
\section{Definición}
\chapter{Función Digamma}
\section{Definición}
\chapter{Función Zeta}
\section{Definición}
\chapter{Función Eta de Dirichilet}
\section{Definición}
\chapter{Función Polygamma}
\section{Definición}
\chapter{Función Polylogarithm}
\section{Definición}
\chapter{Función Hypergemetric Ordinaria}
\section{Definición}
\chapter{Función Error}
\section{Definición}
\chapter{Función  Integral Exponencial  }
\section{Definición}
\chapter{Integral elítica completa  }
\section{Definición}
\chapter{Función Seno Integral}
\section{Definición}
\chapter{Función Coseno Integral}
\section{Definición}
\chapter{Integral involucrando Coseno Integral y Seno Integral}
\section{Definición}
\chapter{Función Logaritmica Integral}
\section{Definición}
\chapter{Función Clausen}
\section{Definición}
\chapter{Función Integral de Clausen}
\section{Definición}
\chapter{Función G de Barnes}
\section{Definición}
%\input{class/ejemplo.tex}
\end{document}
